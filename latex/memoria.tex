\setcounter{chapter}{9}

\chapter{Diseño del sistema}
El diseño del sistema se basa en cuatro pilares: sensórica ambiental, alimentación y autonomía, almacenamiento local y conexión inalámbrica. El objetivo del prototipo es adquirir medidas de forma periódica y fiable, conservar los datos localmente durante un mínimo de dos semanas y permitir posteriormente su volcado a un dispositivo externo. Estas metas se alinean con los requerimientos del proyecto y condicionan el diseño \textit{hardware} y la arquitectura del \textit{firmware} que se describen en los capítulos siguientes.

\section{Arquitectura del sistema}
La Figura \ref{fig:block_diagram} muestra un diagrama conceptual del sistema.

\begin{figure}[H]
    \centering
    \includegraphics[width=1\linewidth]{imgs/diag/block_diagram.png}
    \caption{\textit{Diagrama de bloques conceptual del sistema. Elaboración propia.}}
    \label{fig:block_diagram}
\end{figure}

\section{Decisiones de diseño}
En este apartado se recogen las decisiones de alto nivel que definen el comportamiento del sistema. Estas elecciones condicionan tanto el diseño \textit{hardware} como la arquitectura del \textit{firmware} que se desarrollan en los capítulos siguientes.

\subsection{Periodo de muestreo}
El periodo de muestreo se ha fijado en 20 minutos debido a que ofrece un equilibrio razonable entre resolución temporal y autonomía del sistema. En una lanza meteorológica no es necesario muestrear continuamente, pero tampoco interesa espaciar demasiado las medidas si luego se quiere analizar la evolución diaria o detectar cambios relativamente rápidos. Con 20 minutos se obtiene una serie temporal suficientemente densa para el estudio posterior, manteniendo el tiempo activo bajo control y permitiendo que la mayor parte del tiempo el nodo permanezca en bajo consumo.

\subsection{Almacenamiento local robusto y simple}
Una decisión clave del diseño es que los datos se almacenen localmente, de forma que la recolección no dependa de tener conectividad en el momento de medir. El objetivo mínimo es poder conservar al menos dos semanas de histórico; con una muestra cada 20 minutos esto supone: $2~\mathrm{semanas}\times7~\mathrm{días}\times24~\mathrm{horas}\times3~\mathrm{muestras/h}=1008~\mathrm{muestras}$ en 14 días. Para cumplirlo se emplea una memoria no volátil de 32 kB (256 kbits) organizada en slots de tamaño fijo de 32 bytes. Un slot se reserva para cabecera y metadatos del sistema (identificación, puntero de escritura, número de slots ocupados, etc.), mientras que los 1023 slots restantes se dedican a almacenar muestras.

\subsection{Tolerancia a datos inválidos}
El sistema se ha planteado para que la aparición de datos inválidos en una muestra no implique perder el resto de información de ese instante. En condiciones reales es posible que algún sensor devuelva lecturas erróneas o no responda puntualmente; por ello, cada adquisición registra los datos válidos y marca explícitamente los que no lo sean mediante un vector de validez asociado a la muestra. Esta decisión mejora la robustez del registro y facilita el análisis posterior, ya que permite filtrar únicamente los valores inválidos sin romper la continuidad temporal de la serie.

\subsection{Referencia temporal en cada muestra}
Cada registro incluye una referencia temporal (fecha y hora), lo cual es esencial para el análisis posterior, tanto si se comparan variables entre sí como si se correlacionan con otras externas. Para ello se utiliza un reloj en tiempo real (RTC) que mantiene la hora de manera estable, garantizando que cada muestra queda etiquetada con el instante de adquisición.

\subsection{Tensión de trabajo}
Para simplificar la integración y asegurar compatibilidad entre bloques, el diseño se orienta a que la electrónica del sistema funcione a 3.3 V siempre que sea posible. Mantener un único bus de alimentación reduce conversiones de nivel, facilita la selección de sensores y periféricos y evita complejidad innecesaria. Las excepciones naturales a esta regla son los elementos de potencia como batería y paneles fotovoltaicos, que operan a tensiones distintas y se tratan dentro del subsistema de alimentación.

\subsection{Modularidad del sistema}
El diseño se plantea de forma modular permitiendo ampliar el sistema añadiendo sensores y, si alguno falla o no se adapta bien al despliegue, sustituirlo por otro equivalente sin rediseñar todo el conjunto. Para ello se define un nodo central que concentra el control y al que se conectan los diferentes componentes y sensores, de manera que las variaciones se limiten a los módulos afectados.

\subsection{Protocolo de comunicación inalámbrica}
Como canal inalámbrico se propone usar \textit{Bluetooth Low Energy} (BLE) por su bajo consumo y por la facilidad de interacción con dispositivos externos comunes, lo que encaja con el objetivo de realizar un volcado de datos en campo hacia un dron o un portátil. La comunicación se entiende como una función de extracción de información puntual (\textit{dump}), no como transmisión continua. \\

Dentro del alcance del proyecto, la conectividad BLE se considera una funcionalidad complementaria: el sistema base está diseñado para funcionar correctamente midiendo y almacenando localmente aunque el enlace inalámbrico no se complete.

\chapter{Diseño hardware}

\section{Selección de componentes}
Para la selección de los componentes se tuvo especialmente en cuenta la disponibilidad de placas de evaluación de estos, para facilitar la realización de pruebas de concepto y validación temprana del sistema (véase la Sección \ref{seccion:primer_test}).

\subsection{Gestión de potencia (PMIC)}
Para el subsistema de alimentación se ha seleccionado el \textbf{AEM10941} de e-peas \cite{aem_datasheet}, un PMIC orientado a \textit{energy harvesting}. Este tipo de circuito permite alimentar el sistema a partir de una fuente o \textit{source} (en este caso paneles fotovoltaicos) y gestionar simultáneamente la carga de una batería de almacenamiento, maximizando la captación de energía y garantizando una salida regulada estable.

\begin{figure}[H]
    \centering
    \begin{minipage}{0.6\textwidth}
        \centering
        \includegraphics[width=0.75\linewidth]{imgs/comp/aem10941.png}
    \end{minipage}%
    \begin{minipage}{0.4\textwidth}
        \centering
        \includegraphics[width=0.75\linewidth]{imgs/comp/aem10941_dev.png}
    \end{minipage}
    \caption{\textit{Chip AEM10941 en formato QFN y su respectivo módulo de evaluación. e-peas.}}
    \label{fig:aem10941}
\end{figure}

La elección del AEM10941 encaja especialmente bien con el contexto de una lanza meteorológica autónoma porque integra en un único chip: la gestión de entrada de los paneles solares, la carga y control del elemento de almacenamiento, y las salidas reguladas necesarias para alimentar la electrónica. Con ello se reducen los componentes y reguladores necesarios. \\

En cuanto a su funcionamiento, el AEM10941 dispone de un puerto de entrada para la fuente de \textit{harvesting} donde se conectan los paneles solares. Además, incorpora dos puertos para elementos de almacenamiento (baterías o supercondensadors): uno principal y otro secundario. En el prototipo se utiliza únicamente el puerto principal, ya que la batería de \textit{back-up} se conecta directamente a la MCU (pin \texttt{VBAT}) y se emplea exclusivamente para mantener los servicios mínimos del microcontrolador, principalmente el RTC y el dominio de respaldo. \\

El circuito ofrece dos salidas reguladas, \texttt{HVOUT} (\textit{High Voltage Output}) y \texttt{LVOUT} (\textit{Low Voltage Output}). Esto permite alimentar cargas con diferentes necesidades de tensión o separar dominios de alimentación. En este diseño, como toda la electrónica se orienta a operar a 3.3 V, se utiliza únicamente \texttt{HVOUT} como raíl de alimentación y se prescinde de \texttt{LVOUT}, simplificando la distribución de potencia.

\begin{figure}[H]
    \centering
    \includegraphics[width=0.75\linewidth]{imgs/aem_op_diagram.png}
    \caption{\textit{Diagrama de estados del AEM10941. e-peas.}}
    \label{fig:aem_op_diagram}
\end{figure}

Una ventaja adicional del AEM10941 es la presencia de pines de \textit{status} que indican el modo de funcionamiento: disponibilidad de energía de entrada, estado de carga, habilitación de salidas y condiciones de alimentación insuficiente. Estas señales permiten que la MCU adapte su comportamiento y entre en bajo consumo cuando el sistema se encuentra en un estado energético crítico.

\subsection{Monitoreo de potencia}
Para monitorizar el estado energético del sistema se ha optado por el \textbf{INA3221} de Texas Instruments \cite{ina_datasheet}, un chip integrado que incorpora tres canales independientes para mediciones de tensión y corriente mediante resistencias \textit{shunt}. El uso de tres canales permite monitorear simultáneamente los puntos clave del sistema: un canal dedicado a los paneles fotovoltaicos (entrada de energía), otro canal dedicado a la batería (estado de almacenamiento) y un tercer canal dedicado al consumo del sistema (MCU + sensores + periféricos), medido sobre el bus principal de alimentación. Con esta información se puede estimar la potencia instantánea y caracterizar el consumo medio en condiciones reales.

\begin{figure}[H]
    \centering
    \begin{minipage}{0.55\textwidth}
        \centering
        \includegraphics[width=0.9\linewidth]{imgs/comp/ina3221.png}
    \end{minipage}%
    \begin{minipage}{0.45\textwidth}
        \centering
        \includegraphics[width=0.75\linewidth]{imgs/comp/ina3221_dev.png}
    \end{minipage}
    \caption{\textit{Chip INA3221 en formato QFN y su respectivo módulo de evaluación. Texas Instruments.}}
    \label{fig:ina3221}
\end{figure}

Además, el INA3221 dispone de pines de alerta configurables que pueden usarse para notificar a la MCU si una magnitud supera un umbral. Esto resulta muy útil para proteger el sistema ante situaciones como un consumo próximo al máximo que el PMIC puede suministrar o desviaciones de tensión que indiquen un estado de batería baja o una condición anómala.

\subsection{Almacenamiento}
Para el almacenamiento local de datos se ha escogido una memoria no volátil de tipo \textbf{FRAM} (\textit{Ferroelectric} RAM), concretamente el modelo \textbf{MB85RS256B} de 256 kbits (32 kB) \cite{fram_datasheet}.

\begin{figure}[H]
    \centering
    \includegraphics[width=0.35\linewidth]{imgs/comp/fram.png}
    \caption{\textit{Chip MB85RS256B en formato SOIC-8. Fujitsu.}}
    \label{fig:fram}
\end{figure}

Existen otras alternativas de memorias no volátiles, como la Flash o la EEPROM, pero presentan inconvenientes para este caso de uso. En memorias Flash la escritura suele estar ligada a una gestión por páginas, lo que implica operaciones de borrado previo y reescritura por bloques; esto añade complejidad al \textit{firmware} y no se ajusta bien a un registro periódico de tamaño pequeño. La EEPROM se adapta mejor a escrituras granulares, pero típicamente ofrece una menor velocidad de escritura y una vida útil más limitada cuando se realizan escrituras repetidas durante largos periodos. La FRAM, en cambio, permite escritura directa sin ciclos de borrado, con baja latencia y una resistencia muy alta a ciclos de escritura, lo que simplifica la implementación del registro y mejora la robustez del sistema a largo plazo.

\begin{table}[H]
    \centering
    \footnotesize
    \begin{tabular}{|l|m{0.15\linewidth}|m{0.15\linewidth}|l|}
        \hline
        \rowcolor{LightSteelBlue1}
         & \textbf{FRAM} & \textbf{EEPROM} & \textbf{Flash} \\
        \hline
        Escritura & Directa & Directa & Requiere borrado por páginas \\
        \hline
        Velocidad de escritura & Alta & Media & Media \\
        \hline
        Ciclos de escritura típicos & $10^{12}$ a $10^{14}$ & $10^{5}$ a $10^{6}$ & $10^{4}$ a $10^{5}$ \\
        \hline
        Energía en escritura & Baja & Alta & Alta \\
        \hline
        Granularidad de actualización & Byte & Byte & Página \\
        \hline
        Capacidad típica disponible & KB a pocos MB & KB a MB & MB a GB \\
        \hline
        Complejidad en firmware & Muy baja & Media & Alta \\
        \hline
        Coste por bit & Alto & Medio & Bajo \\
        \hline
    \end{tabular}
    \normalsize
\caption{\textit{Comparación entre tecnologías de memoria no volátil. Elaboración propia.}}
\label{tab:comparacion_mem}
\end{table}

Como podemos observar en la Tabla \ref{tab:comparacion_mem}, la FRAM también presenta algunos defectos, como poca capacidad de almacenamiento disponible en el mercado y un  mayor coste por bit frente a sus alternativas. Aun así, para este prototipo no hace falta una capacidad enorme: interesa más que sea fiable y que aguante bien el uso repetitivo de escritura, y por eso se ha priorizado la FRAM frente a otras alternativas.

\subsection{Sensor de irradiancia}
La irradiancia solar (W/m²) se mide idealmente con un piranómetro; no obstante su coste es elevado y se reserva normalmente para instrumentación de alta precisión. Dado que el objetivo del prototipo no es realizar una medida certificada sino capturar tendencias y obtener una estimación útil, se opta por usar una aproximación basada en un sensor óptico y una conversión experimental.

\begin{figure}[H]
    \centering
    \begin{minipage}{0.5\textwidth}
        \centering
        \includegraphics[width=0.75\linewidth]{imgs/comp/tsl2591.png}
    \end{minipage}%
    \begin{minipage}{0.5\textwidth}
        \centering
        \includegraphics[width=0.75\linewidth]{imgs/comp/tsl2591_dev.png}
    \end{minipage}
    \caption{\textit{Chip TSL2591 en formato QFN y su respectivo módulo de evaluación. Adafruit.}}
    \label{fig:tsl2591}
\end{figure}

Se ha seleccionado el sensor \textbf{TSL2591} \cite{tsl_datasheet}, capaz de medir luz visible e infrarroja. A partir de ambas componentes se estima la iluminancia en \textit{lux}. Los \textit{lux} representan el flujo luminoso por unidad de área ponderado por la sensibilidad del ojo humano; por tanto, no equivalen directamente a irradiancia física (W/m²). Sin embargo, mediante calibración en condiciones representativas es posible ajustar una relación empírica para aproximar irradiancia a partir de \textit{lux}. Esta solución sacrifica precisión absoluta frente a un piranómetro, pero permite capturar tendencias y niveles aproximados con un coste mucho menor.

\begin{figure}[H]
    \centering
    \includegraphics[width=0.8\linewidth]{imgs/grafica_lux.png}
    \caption{\textit{Ponderación espectral utilizada para el cálculo de iluminancia (lux) en función de la sensibilidad del ojo humano. EVOLUX.} \cite{lux-vs-wm2}}
    \label{fig:grafica_lux}
\end{figure}

\subsection{Sensor de temperatura y humedad del aire}
La temperatura y la humedad relativa del aire se miden con un único sensor para reducir complejidad y consumo. Se ha seleccionado el chip \textbf{SHT31} \cite{sht_datasheet}, que ofrece un equilibrio adecuado entre coste, estabilidad y prestaciones dentro de su familia.

\begin{figure}[H]
    \centering
    \begin{minipage}{0.5\textwidth}
        \centering
        \includegraphics[width=0.75\linewidth]{imgs/comp/sht31.png}
    \end{minipage}%
    \begin{minipage}{0.5\textwidth}
        \centering
        \includegraphics[width=0.75\linewidth]{imgs/comp/sht31_dev.png}
    \end{minipage}
    \caption{\textit{Chip SHT31 en formato QFN y su respectivo módulo de evaluación. Sensirion.}}
    \label{fig:sht31}
\end{figure}

El SHT31, adicionalmente, incorpora un calentador interno (\textit{heater}) que puede activarse puntualmente en condiciones de humedad elevada o presencia de rocío. Esta función ayuda a mitigar problemas de condensación sobre el sensor y mejora la recuperación en situaciones ambientales extremas.

\subsection{Sensor de temperatura del terreno}
Para el suelo se emplean sensores separados de temperatura y humedad, ya que los sensores combinados asequibles suelen tener limitaciones en precisión, estabilidad o encapsulado. La temperatura del terreno se mide con un \textbf{DS18B20} \cite{dfr_datasheet} en su versión impermeable: el módulo \textbf{DFR0198}. Este encapsulado permite el uso directo sobre el terreno sin degradación por humedad y simplifica el despliegue al venir protegido mecánicamente.

\begin{figure}[H]
    \centering
    \begin{minipage}{0.5\textwidth}
        \centering
        \includegraphics[width=0.75\linewidth]{imgs/comp/dfr0198.png}
    \end{minipage}%
    \begin{minipage}{0.5\textwidth}
        \centering
        \includegraphics[width=0.75\linewidth]{imgs/comp/ds18b20.png}
    \end{minipage}
    \caption{\textit{Módulo impermeable DFR0198 y sensor DS18B20 en formato TO-92. DFRobot.}}
    \label{fig:dfr0198}
\end{figure}

\subsection{Sensor de humedad del terreno}
Para medir la humedad del terreno se ha elegido el \textbf{SEN0308} \cite{sen_datasheet}, un sensor de tipo capacitivo. Se ha preferido este tipo frente a los sensores resistivos más comunes, porque en la práctica estos últimos suelen dar más problemas cuando se dejan enterrados durante mucho tiempo. 

\begin{figure}[H]
    \centering
    \begin{minipage}{0.5\textwidth}
        \centering
        \includegraphics[width=0.9\linewidth]{imgs/comp/sen0308.png}
    \end{minipage}%
    \begin{minipage}{0.5\textwidth}
        \centering
        \includegraphics[width=0.9\linewidth]{imgs/comp/sen0308__2.png}
    \end{minipage}
    \caption{\textit{Sensor capacitivo SEN0308. DFRobot.}}
    \label{fig:sen0308}
\end{figure}

Un sensor resistivo mide la humedad haciendo pasar una señal eléctrica entre dos partes metálicas en contacto con la tierra. Con el uso, ese contacto eléctrico con el suelo puede provocar deterioro y corrosión de los electrodos. Esto hace que, tras semanas o meses, el sensor no solo sea menos fiable, sino que también sea difícil comparar medidas entre distintos sensores o campañas. \\

El SEN0308, en cambio, al ser capacitivo no depende de que el suelo conduzca corriente de la misma manera, sino que detecta cambios eléctricos asociados al contenido de agua y, por eso, suele ser más estable para despliegues prolongados y con poco mantenimiento. Aun así, su lectura no depende únicamente de la humedad: factores como la salinidad y la composición del suelo también influyen en la medida. Por ello conviene realizar una calibración básica en el entorno real para interpretar la señal de manera consistente.

\subsection{Microcontrolador}
Por último, como MCU se ha seleccionado el \textbf{STM32WB55RG} de STMicroelectronics \cite{stm32_datasheet} \cite{stm32wb55rg}. Este microcontrolador pertenece a la familia STM32, muy extendida en sistemas embebidos por su buen equilibrio entre prestaciones, consumo y ecosistema de desarrollo. Entre las ventajas generales de la gama \textit{STM32} destacan la amplia documentación y notas de aplicación, la disponibilidad de placas de evaluación, como la \textbf{NUCLEO-WB55RG} \cite{nucleo-wb55rg}, y el soporte de herramientas profesionales como STM32CubeMX y STM32CubeIDE, que simplifican la configuración de periféricos y aceleran el desarrollo del \textit{firmware}.

\begin{figure}[H]
    \centering
    \begin{minipage}{0.45\textwidth}
        \centering
        \includegraphics[width=0.75\linewidth]{imgs/comp/stm32wb55rg.png}
    \end{minipage}%
    \begin{minipage}{0.55\textwidth}
        \centering
        \includegraphics[width=0.75\linewidth]{imgs/comp/nucleo_wb55rg.png}
    \end{minipage}
    \caption{\textit{Chip STM32WB55RG en formato QFN y placa de evaluación y desarrollo NUCLEO-WB55RG. STMicroelectronics.}}
    \label{fig:stm32}
\end{figure}

La elección de este modelo concreto responde además a varios requisitos del prototipo:

\begin{itemize}
    \item \textbf{Muy bajo consumo:} el STM32WB ofrece modos de bajo consumo adecuados para el funcionamiento, permitiendo reducir significativamente el gasto energético entre muestras.
    \item \textbf{BLE integrado:} incorpora radio y conectividad \textit{Bluetooth Low Energy} sin necesidad de módulos externos, reduciendo así complejidad y número de componentes.
    \item \textbf{Periféricos y pines suficientes:} esta variante proporciona un encapsulado de 68 pines y dispone de interfaces habituales (I2C, SPI, UART, GPIO y ADC), lo que permite integrar todos los componentes sin recurrir a soluciones adicionales.
    \item \textbf{Capacidad de proceso:} ofrece potencia de cálculo y memoria suficientes para ejecutar la lógica de adquisición, filtrado y empaquetado de muestras, gestión de memoria y control de estados energéticos del sistema.
\end{itemize}

\begin{figure}[H]
    \centering
    \includegraphics[width=1\linewidth]{imgs/stm32wbxxxx.png}
    \caption{\textit{Comparación entre diferentes modelos del STM32WB. STMicroelectronics.} \cite{stm32wb55rg}}
    \label{fig:stm32wbxxxx}
\end{figure}

Una característica diferencial del STM32WB55 es su arquitectura \textit{dual-core}. Integra un núcleo ARM Cortex-M4, orientado a ejecutar la aplicación principal, y un núcleo ARM Cortex-M0+ que actúa como coprocesador de comunicaciones. \\

En particular, el \textit{stack} de \textit{Bluetooth Low Energy} y la gestión de la radio se ejecutan en el Cortex-M0+. Esto permite separar claramente dos dominios: por un lado la aplicación (M4) y por otro las tareas de comunicación inalámbrica (M0+), que requieren temporizaciones estrictas y gestión específica del transceptor. Esta separación facilita que la comunicación BLE no interfiera con la lógica del sistema y, al mismo tiempo, contribuye a un funcionamiento más robusto y eficiente energéticamente, al delegar la pila inalámbrica en el núcleo dedicado.

\section{Arquitectura del \textit{hardware}}

\begin{figure}[H]
    \centering
    \includegraphics[width=1\linewidth]{imgs/diag/hw_arch_diagram.png}
    \caption{\textit{Diagrama de la arquitectura hardware del sistema. Elaboración propia.}}
    \label{fig:hw_arch_diagram}
\end{figure}

En la Figura \ref{fig:hw_arch_diagram} observamos cómo se estructuran los principales componentes y los diferentes bloques funcionales mencionados anteriormente: sensores meteorológicos (verde), control y gestión de potencia (naranja), almacenamiento local (morado) y conexión inalámbrica (amarillo). En el centro se sitúa la MCU (gris), que actúa como nodo principal del sistema. \\

Dentro del microcontrolador se distinguen sus dos núcleos (azul), el Cortex-M4 y el Cortex-M0+. El primero ejecuta la aplicación principal y se comunica con el resto de periféricos, mientras que el segundo se encarga de la gestión de la pila de \textit{Bluetooth Low Energy} (BLE \textit{stack}). Ambos núcleos están comunicados internamente, permitiendo que la conectividad inalámbrica no interfiera con la lógica principal del sistema.

\subsection{Interfaces y protocolos}

\begin{figure}[H]
    \centering
    \includegraphics[width=1\linewidth]{imgs/diag/protocol_diagram.png}
    \caption{\textit{Diagrama de las interfaces y protocolos de comunicación del sistema. Elaboración propia.}}
    \label{fig:protocol_diagram}
\end{figure}

La Figura \ref{fig:protocol_diagram} detalla las interfaces y protocolos utilizados para la interconexión entre los diferentes componentes del sistema: I2C, SPI, UART, SWD, 1-Wire, ADC y BLE. \\

El bus \textbf{I2C} se emplea como interfaz principal para sensores y monitorización, ya que permite conectar múltiples periféricos compartiendo las mismas líneas (\texttt{SDA} y \texttt{SCL}) y reduce el número de pines necesarios en la MCU. Por este motivo se ha priorizado su uso siempre que el dispositivo lo soportaba. \\

La memoria FRAM se comunica mediante \textbf{SPI}. Aunque existen alternativas habituales como I2C o buses paralelos, se ha optado por SPI por su equilibrio entre una mayor velocidad que I2C y una integración más sencilla que un bus paralelo (que requeriría más pines y complica el ruteado). En este enlace se utilizan \texttt{MOSI}, \texttt{MISO} y \texttt{SCK}, junto con \texttt{CS} para la selección del periférico y una línea adicional (\texttt{WP}) para poder escribir en registros protegidos. \\

El sensor DFR0198 (o DS18B20) utiliza el protocolo \textbf{1-Wire}, menos común en este tipo de sistemas pero ventajoso al requerir una única línea de datos para comunicación. Por su parte, el sensor SEN0308 proporciona una salida analógica, que se adquiere mediante el \textbf{ADC} integrado en el STM32. \\

Para tareas de desarrollo y validación se incluye una interfaz \textbf{UART} hacia el PC, utilizada como consola serie por su sencillez y amplia compatibilidad. La programación del microcontrolador se realiza mediante \textbf{SWD} (\textit{Serial Wire Debug}), utilizando un programador externo ST-LINK V2. Finalmente, la comunicación inalámbrica del sistema se implementa mediante \textbf{BLE}. \\

Además de los buses principales, aparecen señales digitales auxiliares como \textit{Alert} y \textit{Status}, conectadas a entradas de interrupción (\textbf{EXTI}) de la MCU. Estas líneas permiten notificar eventos relevantes sin necesidad de sondeo continuo, reduciendo el consumo y simplificando la lógica de control.

\section{Prueba de concepto} \label{seccion:primer_test}

Antes de abordar el diseño de la PCB final se realizó una fase de verificación incremental del sistema mediante una prueba de concepto. El objetivo de esta etapa fue reducir riesgos de integración y validar el funcionamiento de los componentes principales en condiciones controladas, asegurando que tanto la alimentación como la adquisición de datos y la comunicación inalámbrica eran viables con los componentes seleccionados. \\

La primera aproximación consistió en adquirir placas de evaluación de los elementos \textit{core} del sistema: la NUCLEO-WB55RG como plataforma de desarrollo para el microcontrolador, y módulos para el AEM10941, INA3221 y TSL2591. Con estas placas, junto con baterías, paneles fotovoltaicos, cableado tipo \textit{jumper} y \textit{protoboards}, se montó un circuito de pruebas que replicaba el funcionamiento básico del sistema final: gestión de energía mediante el PMIC, mediciones de consumo con el monitor de potencia y lectura de sensores desde la MCU. Esta configuración permitió comprobar de forma temprana aspectos críticos como la compatibilidad de niveles lógicos, el funcionamiento de los buses de comunicación y el comportamiento general del sistema bajo alimentación real.

\begin{figure}[H]
    \centering
    \includegraphics[width=0.8\linewidth]{imgs/photo/circuit_1.jpg}
    \caption{\textit{Primera prueba de concepto del core del sistema con placas de evaluación, cableado tipo jumper y protoboards. Elaboración propia.}}
    \label{fig:circuit_1}
\end{figure}

Una vez verificado el correcto funcionamiento del núcleo del sistema, el siguiente paso fue desarrollar un primer prototipo de PCB fabricado mediante fresado de cobre, utilizando las facilidades del departamento de ESAII de la UPC. Este método de prototipado consiste en coger una placa aislante recubierta de cobre y emplear una fresadora CNC para eliminar mecánicamente el cobre de la superficie, dejando definidas las pistas, \textit{pads} y contornos del circuito. A diferencia de una PCB industrial (fabricada por procesos fotoquímicos), el fresado permite iteraciones rápidas en laboratorio, pero impone restricciones geométricas y de fabricación que condicionan el diseño.

\begin{figure}[H]
    \centering
    \begin{minipage}{0.65\textwidth}
        \centering
        \includegraphics[width=0.8\linewidth]{imgs/photo/fresadora_1.jpg}
    \end{minipage}%
    \begin{minipage}{0.35\textwidth}
        \centering
        \includegraphics[width=0.85\linewidth]{imgs/photo/fresadora_2.jpg}
    \end{minipage}
    \caption{\textit{Imágenes de la fresadora de cobre utilizada. Elaboración propia.}}
    \label{fig:fresadora}
\end{figure}

En este proyecto en concreto, la PCB fresada se limitó a una de dos capas (\textit{top} y \textit{bottom}) y de dimensiones más conservadoras que las típicas en una fabricación profesional, se trabajó con anchos de pista y separaciones mínimas del orden de 0.5 mm, y con taladros mínimos alrededor de 0.85 mm. Durante esta fase aparecieron algunos errores de diseño como problemas con la unión de planos GND.

\begin{figure}[H]
    \centering
    \includegraphics[width=0.75\linewidth]{imgs/photo/circuit_2.jpg}
    \caption{\textit{PCB de cobre fresado del prototipo. Elaboración propia.}}
    \label{fig:circuit_2}
\end{figure}

\begin{figure}[H]
    \centering
    \includegraphics[width=0.75\linewidth]{imgs/photo/circuit_3.jpg}
    \caption{\textit{PCB de cobre fresado después de soldar conectores y conectar los módulos de evaluación. Elaboración propia.}}
    \label{fig:circuit_3}
\end{figure}

Tras estabilizar el funcionamiento del prototipo y confirmar el comportamiento del \textit{core}, se ampliaron las pruebas incorporando el resto de sensores del sistema. Para ello se adquirieron los módulos de evaluación del SHT31, DFR0198 y SEN0308, se conectaron al sistema y se verificó la correcta lectura desde la MCU, así como su integración con la lógica de adquisición.

\begin{figure}[H]
    \centering
    \includegraphics[width=0.75\linewidth]{imgs/photo/circuit_4.jpg}
    \caption{\textit{Prueba de concepto final del core conectada. Elaboración propia.}}
    \label{fig:circuit_4}
\end{figure}

\begin{figure}[H]
    \centering
    \includegraphics[width=0.9\linewidth]{imgs/photo/circuit_5.jpg}
    \caption{\textit{Prueba de concepto completa conectada y imprimiendo logs por consola. Elaboración propia.}}
    \label{fig:circuit_5}
\end{figure}

Con el conjunto de pruebas satisfactorio, se dio por validada la arquitectura y se procedió al diseño e integración de la PCB final, ya con un nivel de confianza mayor y con criterios de diseño refinados a partir de los aprendizajes obtenidos en la plataforma de testeo.

\newpage

\section{Esquemáticos}

\subsection{Potencia y alimentación}

\subsubsection{AEM10941}
La Figura \ref{fig:aem_block_diagram} muestra el diagrama de bloques de referencia que nos proporciona el \textit{datasheet} del fabricante, a partir del cual se ha desarrollado el esquemático del circuito del AEM10941. En ella vemos los pines principales: \texttt{SRC}, \texttt{BATT} y \texttt{HVOUT}, que indican la entrada de los paneles solares, la batería y la salida de 3.3 V regulada, respectivamente.

\begin{figure}[H]
    \centering
    \includegraphics[width=1\linewidth]{imgs/aem_block_diagram.png}
    \caption{\textit{Diagrama de bloques funcional de referencia del AEM10941. e-peas.}}
    \label{fig:aem_block_diagram}
\end{figure}

En nuestro sistema solo se utiliza la salida \texttt{HVOUT} y no \texttt{LVOUT} por lo que lo pines de \textit{enable} de los LDOs se fijan a \texttt{ENVH = HIGH} y \texttt{ENVL = LOW} (el voltaje lógico de \texttt{HIGH} es \texttt{VBUCK}). Además la salida \texttt{LVOUT} se deja flotando. \\

La configuración de la batería y los voltajes de salida de los LDOs se establecen mediante los pines de \texttt{CFG[2:0]}, fijados a \texttt{111} en binario, lo que selecciona el perfil de carga correspondiente a una batería Li-ion. En la Figura \ref{fig:aem_cfg} se observa la tabla de posibles configuraciones. Como no se emplea un modo de configuración personalizado, los pines \texttt{SET\_OVDIS}, \texttt{SET\_CHRDY}, \texttt{SET\_OVCH} y \texttt{FB\_HV} no se utilizan y se dejan flotando.

\begin{figure}[H]
    \centering
    \includegraphics[width=0.9\linewidth]{imgs/aem_cfg.png}
    \caption{\textit{Tabla de modos de batería y voltajes de salida de los LDOs del AEM10941. e-peas.}}
    \label{fig:aem_cfg}
\end{figure}

Respecto al MPPT, se selecciona \texttt{SELMPP[1:0] = 00}, que fija el seguimiento al 70\% de la tensión en circuito abierto del panel. Esta elección ofrece un compromiso adecuado entre simplicidad y rendimiento, ya que aproxima el punto de máxima potencia de muchos paneles pequeños en condiciones variables sin necesidad de calibración adicional. En la Figura \ref{fig:aem_mppt} se muestra la tabla de posibles configuraciones del MPPT.

\begin{figure}[H]
    \centering
    \includegraphics[width=0.4\linewidth]{imgs/aem_mppt.png}
    \caption{\textit{Tabla de posibles configuraciones del MPPT del AEM10941. e-peas.}}
    \label{fig:aem_mppt}
\end{figure}

Otro punto punto a tener en cuenta es el pin \texttt{BATT}, que no debe quedar nunca flotante. Dado que se contempla el uso de una batería recambiable, existe la posibilidad de desconexión temporal; por eso se añade un capacitor de 150 µF en el nodo de batería, tal y como recomienda el fabricante. Asimismo, como no se utiliza batería secundaria en este diseño, los pines \texttt{PRIM}, \texttt{FB\_PRIM\_U} y \texttt{FB\_PRIM\_D} se conectan a GND. \\

Los demás condensadores e inductores son los que recomienda el fabricante por defecto. Las señales \texttt{STATUS[2:0]} se conectan directamente al microcontrolador. \\

En la Figura \ref{fig:sch_aem} se expone el esquemático final del subsistema de gestión de potencia.

\begin{figure}[H]
    \centering
    \includegraphics[width=1\linewidth]{imgs/sch/sch_aem.png}
    \caption{\textit{Esquemático final del subsistema de gestión de potencia. Elaboración propia.}}
    \label{fig:sch_aem}
\end{figure}

\subsubsection{INA3221}
La Figura \ref{fig:ina_block_diagram} muestra el diagrama de conexiones recomendado por el fabricante del INA3221. \\

\begin{figure}[H]
    \centering
    \includegraphics[width=0.75\linewidth]{imgs/ina_block_diagram.png}
    \caption{\textit{Diagrama de bloques funcional de referencia del INA3221. Texas Instruments.}}
    \label{fig:ina_block_diagram}
\end{figure}

El INA3221 mide, para cada canal, dos magnitudes: la tensión de bus (diferencia de voltaje entre \texttt{IN+} y \texttt{GND}), y la tensión de la \textit{shunt} (diferencia de voltaje entre \texttt{IN+} e \texttt{IN-}). A partir de esta caída diferencial y del valor de la resistencia \textit{shunt} se puede calcular la corriente del canal. \\

La selección del valor de la resistencia de \textit{shunt} se realiza a partir de la caída máxima deseada y la corriente máxima esperada. Consideramos una caída máxima en la \textit{shunt} de $V_\mathrm{shunt\_max} = 33~\mathrm{mV}$ y una corriente máxima de $I_\mathrm{max} = 150~\mathrm{mA}$:

\[
    R_\mathrm{shunt}=\frac{V_\mathrm{shunt\_max}}{I_\mathrm{max}}=\frac{0.033}{0.15}=0.22~\Omega
\]

\[
    P_\mathrm{shunt}=I_\mathrm{max}^2 \cdot R_\mathrm{shunt}=0.15^2 \cdot 0.22=0.00495~\mathrm{W}
\]

Esta configuración ofrece una precisión aceptable a cambio de una caída de tensión muy pequeña. La disipación de calor también es mínima. En cuanto a la resolución, el INA3221 presenta una cuantización típica de $40~\mathrm{\mu V/bit}$ en el registro de tensión de los canales. Por tanto, la resolución de corriente equivalente quedaría:

\[
    I_\mathrm{LSB}=\frac{40~\mathrm{\mu V/bit}}{R_\mathrm{shunt}}=\frac{40 \cdot 10^{-6}}{0.22}=0.182~\mathrm{mA/bit}
\]

Para reducir ruido de alta frecuencia y mejorar la estabilidad de medida, el fabricante recomienda implementar un filtro de entrada en cada canal que consiste de dos resistencia en serie de 10 $\Omega$ y un condensador de 100 nF entre \texttt{IN+} e \texttt{IN-}. Esta red actúa como filtro pasa-bajo y ayuda a mitigar picos y transitorios. En la Figura \ref{fig:ina_filter} vemos el diagrama de conexión propuesto.

\begin{figure}[H]
    \centering
    \includegraphics[width=0.75\linewidth]{imgs/ina_filter.png}
    \caption{\textit{Red de filtrado recomendada para cada canal del INA3221.}}
    \label{fig:ina_filter}
\end{figure}

El INA3221 incluye además pines de alerta configurables: \texttt{PV}, \texttt{CRITICAL}, \texttt{WARNING} y \texttt{TC}. Este último no lo usaremos, por lo que lo dejamos flotante. Los demás los conectamos a una resistencia de \textit{pull-up} de 10 k$\Omega$ y al STM32. El INA3221 tiene además la posibilidad de escoger entre 4 direcciones I2C mediante el pin \texttt{A0}. Como usaremos la dirección por defecto lo conectamos a GND. \\

En la Figura \ref{fig:sch_ina} se expone el esquemático final del subsistema de monitoreo de potencia.

\begin{figure}[H]
    \centering
    \includegraphics[width=1\linewidth]{imgs/sch/sch_ina.png}
    \caption{\textit{Esquemático final del subsistema de monitoreo de potencia. Elaboración propia.}}
    \label{fig:sch_ina}
\end{figure}

\subsubsection{Bus de 3.3 V} \label{seccion:bus_3V3}
El bus de 3.3 V está diseñado para alimentarse por defecto desde el pin \texttt{HVOUT} del AEM10941, que actúa como fuente principal del sistema durante operación normal. Sin embargo, para facilitar la programación, depuración y pruebas en laboratorio, se incorpora la posibilidad de alimentar dicho bus desde un conector externo (\texttt{3V3\_EXT}). Para seleccionar el origen de la alimentación se incluye un \textit{jumper}, que permite conmutar entre la fuente generada por el PMIC y la alimentación externa.
\begin{figure}[H]
    \centering
    \includegraphics[width=0.45\linewidth]{imgs/sch/sch_bus_3V3.png}
    \caption{\textit{Esquemático de la selección de fuente del bus de 3.3 V. Elaboración propia.}}
    \label{fig:sch_bus_3V3}
\end{figure}

\subsubsection{Alimentación de sensores}
Con el objetivo de minimizar consumo, la alimentación de los sensores se puede activar y desactivar mediante un interruptor de alta implementado con un MOSFET PMOS. El bus de 3.3 V se conecta al source (S) del PMOS, y la alimentación conmutada hacia los sensores al drain (D). La gate (G) se controla desde un pin del STM32.

\begin{figure}[H]
    \centering
    \includegraphics[width=0.5\linewidth]{imgs/sch/sch_sensor_supply_switch.png}
    \caption{\textit{Esquemático del interruptor para alimentación de sensores. Elaboración propia.}}
    \label{fig:sch_sensor_supply_switch}
\end{figure}

Para asegurar un estado seguro por defecto, se añade una resistencia de \textit{pull-up} de 100 k$\Omega$ entre el source y la gate, de modo que, si el microcontrolador no está inicializado, la gate queda al mismo potencial que el source ($V_\mathrm{GS}=0$) y el transistor permanece cortado. El valor elevado de esta resistencia reduce la corriente consumida por la red de polarización y facilita al microcontrolador modificar el nivel de gate sin cargas innecesarias.

\begin{table}[H]
    \centering
    \begin{tabular}{|l|l|l|}
        \hline
        GATE flotando & $V_\mathrm{GS}=V_\mathrm{G}-V_\mathrm{S}=0$ & Interruptor apagado \\
        \hline
        GATE a HIGH & $V_\mathrm{GS}=V_\mathrm{G}-V_\mathrm{S}=0$ & Interruptor apagado \\
        \hline
        GATE a LOW & $V_\mathrm{GS}=V_\mathrm{G}-V_\mathrm{S}<0$ & Interruptor encendido \\
        \hline
    \end{tabular}
    \normalsize
    \caption{\textit{Estados lógicos del control de la gate del MOSFET y su efecto sobre la alimentación  de sensores. Elaboración propia.}}
    \label{tabla:pmos}
\end{table}

Se cumplen además estos puntos:
\begin{itemize}
    \item Se utiliza un sólo interruptor para todos los sensores ambientales.
    \item El interruptor sólo afecta a los sensores, no al STM32 ni al INA3221.
    \item El bus de 3.3 V entra al source del MOSFET después de pasar por el INA3221.
\end{itemize}

\newpage

\subsection{MCU}

\subsubsection{STM32WB55RG}
Para reducir el riesgo de errores en una parte crítica del diseño, el esquema se ha construido tomando como referencia la placa NUCLEO-WB55RG y replicando los esquemáticos del fabricante \cite{stm32_sch}, siempre que era posible. Esta aproximación permite apoyarse en una topología ya validada y minimiza incertidumbres típicas del primer diseño de placa. Las Figuras \ref{fig:sch_stm32} y \ref{fig:sch_mcu_supply} muestran el esquemático final de la MCU.

\begin{figure}[H]
    \centering
    \includegraphics[width=1\linewidth]{imgs/sch/sch_stm32.png}
    \caption{\textit{Esquemático del STM32WB55RG y conexiones principales de control y periféricos. Elaboración propia.}}
    \label{fig:sch_stm32}
\end{figure}

\begin{figure}[H]
    \centering
    \includegraphics[width=1\linewidth]{imgs/sch/sch_mcu_supply.png}
    \caption{\textit{Esquemático de alimentación y desacoplo del STM32WB55RG. Elaboración propia.}}
    \label{fig:sch_mcu_supply}
\end{figure}

Para cada pin de alimentación se ha colocado un condensador de 100 nF lo más próximo posible al encapsulado. Además del desacoplo “básico”, el STM32WB incluye dominios con requisitos adicionales, los cuales podemos observar en la Figura \ref{fig:stm32_supply}. En lugar de definir una topología propia, se ha seguido el esquema de referencia de la NUCLEO-WB55RG para estos pines.

\begin{figure}[H]
    \centering
    \begin{minipage}{0.55\textwidth}
        \centering
        \includegraphics[width=1\linewidth]{imgs/stm32_supply.png}
    \end{minipage}%
    \begin{minipage}{0.45\textwidth}
        \centering
        \includegraphics[width=1\linewidth]{imgs/stm32_smps.png}
    \end{minipage}
    \caption{\textit{Recomendaciones de conexión de alimentación y SMPS del STM32WB55RG. STMicroelectronics.}}
    \label{fig:stm32_supply}
\end{figure}

La Figura \ref{fig:stm32_rst_boot} muestra el circuito para las conexiones de los pines de reset (\texttt{NRST}) y de boot (\texttt{BOOT0}) recomendado y del cual se ha inspirado para conectar el propio.

\begin{figure}[H]
    \centering
    \includegraphics[width=0.6\linewidth]{imgs/stm32_rst_boot.png}
    \caption{\textit{Circuito recomendado para los pines \texttt{NRST} y \texttt{BOOT0} en el STM32WB55RG. STMicroelectronics.}}
    \label{fig:stm32_rst_boot}
\end{figure}

En la Tabla \ref{tabla:pinout} se muestra el \textit{pinout} utilizado del STM32WB55RG para este proyecto \cite{stm32_pinout}.

\small
\begin{longtable}{|m{0.25\linewidth}|m{0.25\linewidth}|m{0.25\linewidth}|}
    \hline
    \rowcolor{LightSteelBlue1}
    \textbf{Pin STM32} & \textbf{Etiqueta} & \textbf{Función} \\
    \hline
    VDD & VDD & Alimentación \\
    \hline
    VDDA & VDDA & Alimentación \\
    \hline
    VDDRF & VDDRF & Alimentación \\
    \hline
    VDDSMPS & VDDSMPS & Alimentación \\
    \hline
    VDDUSB & VDDUSB & Alimentación \\
    \hline
    VBAT & VBAT & Alimentación \\
    \hline
    VSS & GND & Tierra \\
    \hline
    VSSRF & GND & Tierra \\
    \hline
    VSSSMPS & GND & Tierra \\
    \hline
    \hline
    NRST & NRST & Reset \\
    \hline
    VLXSMPS & VLXSMPS & Alimentación \\
    \hline
    VFBSMPS & VFBSMPS & Alimentación \\
    \hline
    VREF & VDDA & Referencia de voltaje \\
    \hline
    RF1 & RF1 & RF \\
    \hline
    OSC\_IN & HSE\_IN & OSC\_IN \\
    \hline
    OSC\_OUT & HSE\_OUT & OSC\_OUT \\
    \hline
    PC14 & LSE\_IN & OSC32\_IN \\
    \hline
    PC15 & LSE\_OUT & OSC32\_OUT \\
    \hline
    PH3 & BOOT0 & Boot \\
    \hline
    \hline
    PA0 & PV\_INA & GPIO\_EXTI0 \\
    \hline
    PA1 & CRI\_INA & GPIO\_EXTI1 \\
    \hline
    PA2 & WAR\_INA & GPIO\_EXTI2 \\
    \hline
    PA3 & S0\_AEM & GPIO\_EXTI3 \\
    \hline
    PA4 & S1\_AEM & GPIO\_EXTI4 \\
    \hline
    PA5 & S2\_AEM & GPIO\_Input \\
    \hline
    PA6 & GATE\_SENS & GPIO\_Output \\
    \hline
    PA7 & DFR0198 & GPIO\_Output \\
    \hline
    PA8 & SEN0308 & ADC1\_IN15 \\
    \hline
    PA10 & USER\_LED & GPIO\_Output \\
    \hline
    PA11 & CS\_FRAM & GPIO\_Output \\
    \hline
    PA13 & SWDIO & SWD \\
    \hline
    PA14 & SWCLK & SWD \\
    \hline
    \hline
    PB3 & SWO & SWD \\
    \hline
    PB6 & USART\_RX & USART1\_TX \\
    \hline
    PB7 & USART\_TX & USART1\_RX \\
    \hline
    PB13 & SCK & SPI2\_SCK \\
    \hline
    PB14 & MISO & SPI2\_MISO \\
    \hline
    PB15 & MOSI & SPI2\_MOSI \\
    \hline
    \hline
    PC0 & SCL & I2C3\_SCL \\
    \hline
    PC1 & SDA & I2C3\_SDA \\
    \hline
    PC12 & WP\_FRAM & GPIO\_Output \\
    \hline
    \caption{\textit{Tabla con el pinout utilizado del STM32, su etiqueta y su función. Elaboración propia.}}
    \label{tabla:pinout}
\end{longtable}
\normalsize

\subsubsection{Osciladores}
El STM32WB55RG permite incorporar dos cristales externos: un oscilador de 32 MHz (HSE) y un oscilador de 32.768 kHz (LSE). El HSE es relevante para el funcionamiento fiable del subsistema RF, mientras que el LSE se recomienda para el RTC por ofrecer mayor precisión que los osciladores internos. Por este motivo se han incorporado ambos en el diseño final. \\

La selección del cristal no depende únicamente de su frecuencia nominal, sino también de la capacitancia de carga requerida. La capacitancia efectiva del circuito se calcula mediante la siguiente fórmula \cite{an2867}:

\begin{itemize}
    \item $C_\mathrm{L}$ es la capacitancia de carga especificada por el cristal.
    \item $C_\mathrm{S}$ representa capacitancias parásitas (pistas, pads y entrada del micro).
    \item $C_\mathrm{1}$ y $C_\mathrm{2}$ son los condensadores externos en cada pin del cristal.
\end{itemize}

\[
    C_\mathrm{L}=\frac{C_\mathrm{1}\cdot C_\mathrm{2}}{C_\mathrm{1}+C_\mathrm{2}}+C_\mathrm{S};\qquad C_\mathrm{1}=C_\mathrm{2}=2\cdot(C_\mathrm{L}-C_\mathrm{S})
\]

Se fija $C_\mathrm{S}=3~\mathrm{pF}$. En el caso del HSE, no se montan condensadores externos porque la carga se puede ajustar mediante configuración interna del microcontrolador \cite{an5042}. En cambio, para el LSE sí se seleccionan condensadores externos. El cristal LSE escogido tiene $C_\mathrm{L}=9~\mathrm{pF}$ por lo que el cálculo conduce a:

\[
    C_\mathrm{1}=C_\mathrm{2}=2\cdot(9-3)=12~\mathrm{pF}
\]

\begin{figure}[H]
    \centering
    \includegraphics[width=0.9\linewidth]{imgs/sch/sch_oscilators.png}
    \caption{\textit{Esquemático de los osciladores HSE (32 MHz) y LSE (32.768 kHz) y sus componentes asociados. Elaboración propia.}}
    \label{fig:sch_oscilators}
\end{figure}

\subsubsection{RF}
Es la parte más sensible del diseño, por lo que se ha seguido estrictamente la documentación de referencia y notas de aplicación de STMicroelectronics \cite{an5129} \cite{an6335} \cite{an5165}. \\

Se utiliza el módulo MLPF-WB55-01E3 (recomendado para STM32WB), junto con una red adicional tipo “H” formada por dos condensadores de 1.2 pF y un inductor de 3.6 nH para adaptar la señal entre el pin RF1 del microcontrolador y la antena. Siguiendo la referencia de la NUCLEO-WB55RG, el primer condensador de la red se omite, manteniendo la topología validada por el fabricante.

\begin{figure}[H]
    \centering
    \includegraphics[width=1\linewidth]{imgs/stm32_rf.png}
    \caption{\textit{Topología de referencia del subsistema RF para los STM32WB.}}
    \label{fig:stm32_rf}
\end{figure}

Además, se implementan pistas RF con impedancia controlada. La \textit{Application Note AN6335} \cite{an6335} indica que el tramo entre el STM32 y el MLPF debe aproximarse a 62 $\Omega$ (62R), mientras que el tramo desde el filtro hacia la antena debe ser de 50 $\Omega$ (50R). Estas impedancias serán claves para el dimensionado de las pistas y la selección del \textit{stack-up} de la PCB.

\begin{figure}[H]
    \centering
    \includegraphics[width=1\linewidth]{imgs/sch/sch_rf.png}
    \caption{\textit{Implementación del filtrado y adaptación RF entre RF1, MLPF y antena. Elaboración propia.}}
    \label{fig:sch_rf}
\end{figure}

\subsubsection{FRAM}
La memoria FRAM se conecta mediante bus SPI común. Adicionalmente incorpora dos pines de control: \texttt{HOLD}, que permite pausar temporalmente la comunicación sin perder el estado, y \texttt{WP}, que habilita la protección de escritura. En este diseño no se utiliza la función HOLD, por lo que se conecta a 3.3 V a través de una resistencia de \textit{pull-up}. El pin \texttt{WP} se controla desde la MCU. Finalmente, se incluye un condensador de 100 nF de desacoplo en la alimentación del integrado.

\begin{figure}[H]
    \centering
    \includegraphics[width=0.6\linewidth]{imgs/sch/sch_fram.png}
    \caption{\textit{Conexión de la memoria FRAM. Elaboración propia.}}
    \label{fig:sch_fram}
\end{figure}

\subsubsection{LEDs}
Se integran dos LEDs en el sistema: uno de POWER, que indica la presencia de alimentación, y otro de USER, controlado por un pin del STM32. El LED de usuario se conecta con el cátodo a GND, de modo que se enciende al fijar el pin a HIGH, y el valor de la resistencia se dimensiona para un consumo aproximado de 1 mA, suficiente para que sean visibles sin penalizar el consumo energético. \\

El cálculo se realiza con la expresión:

\[
    I=\frac{V_\mathrm{DD}-V_\mathrm{F}}{R}
\]

Donde $V_\mathrm{DD}=3.3~\mathrm{V}$ y $V_\mathrm{F}$ es la caída de tensión directa del LED, la cual depende del color del mismo.

\begin{figure}[H]
    \centering
    \includegraphics[width=0.35\linewidth]{imgs/sch/sch_leds.png}
    \caption{\textit{Circuito de los LEDs de POWER y USER. Elaboración propia.}}
    \label{fig:sch_leds}
\end{figure}

\subsection{Conectores}

\subsubsection{SWD}
Para la programación del STM32WB55RG se ha optado por la interfaz SWD (Serial Wire Debug), utilizando un programador ST-LINK/V2. El ST-LINK estándar emplea un conector de 20 pines (ver Figura \ref{fig:stlink_swd}), pero para este proyecto dicho conector resulta excesivamente grande para la PCB y, además, la mayoría de sus pines son redundantes o corresponden a GND.

\begin{figure}[H]
    \centering
    \begin{minipage}{0.5\textwidth}
        \centering
        \includegraphics[width=0.85\linewidth]{imgs/swd_pinout.png}
    \end{minipage}%
    \begin{minipage}{0.5\textwidth}
        \centering
        \includegraphics[width=0.85\linewidth]{imgs/stlink_v2.png}
    \end{minipage}
    \caption{\textit{Pinout del conector SWD del ST-LINK y programador ST-LINK/V2 utilizado durante el desarrollo.}}
    \label{fig:stlink_swd}
\end{figure}

Por este motivo, se ha reducido la conexión a las señales mínimas necesarias para programación y depuración: \texttt{Vref}, \texttt{SWDIO}, \texttt{SWCLK}, \texttt{GND}, \texttt{NRST} y, de forma opcional, \texttt{SWO}. Adicionalmente se enruta una opción de alimentación hacia el jumper del bus de 3.3 V descrito en la sección “Bus de 3.3 V” (Sección \ref{seccion:bus_3V3}), de forma que durante pruebas en laboratorio sea posible alimentar el sistema desde una fuente externa o, si el programador lo permite, desde el propio ST-LINK. En cualquier caso, \texttt{Vref} se utiliza únicamente como referencia de tensión, no para alimentación.
 
\begin{figure}[H]
    \centering
    \includegraphics[width=0.35\linewidth]{imgs/sch/sch_swd.png}
    \caption{\textit{Esquemático de la interfaz SWD. Elaboración propia.}}
    \label{fig:sch_swd}
\end{figure}

\subsubsection{Baterias y paneles solares}
La Figura \ref{fig:sch_supply_conn} muestra los conectores asociados a las baterías y paneles solares y sus conexiones. Se han añadido condensadores de desacoplo de 100 nF en las entradas de alimentación por defecto.

\begin{figure}[H]
    \centering
    \includegraphics[width=1\linewidth]{imgs/sch/sch_supply_conn.png}
    \caption{\textit{Esquemático de los conectores de la entrada fotovoltaica y las baterías. Elaboración propia.}}
    \label{fig:sch_supply_conn}
\end{figure}

\subsubsection{Sensores}
En la Figura \ref{fig:sch_sensor_conn} se presentan los conectores utilizados para los diferentes sensores del sistema.

\begin{figure}[H]
    \centering
    \includegraphics[width=1\linewidth]{imgs/sch/sch_sensor_conn.png}
    \caption{\textit{Esquemático de los conectores de los sensores. Elaboración propia.}}
    \label{fig:sch_sensor_conn}
\end{figure}

\subsubsection{Otros conectores}
En la Figura \ref{fig:sch_uart_gnd} se incluyen conectores auxiliares como la conexión UART para imprimir por consola. También se han añadido un par de puntos de GND accesibles, útiles como referencia de medida para herramientas como el multímetro u osciloscopio, simplificando tareas de verificación en laboratorio.

\begin{figure}[H]
    \centering
    \includegraphics[width=0.6\linewidth]{imgs/sch/sch_uart_gnd.png}
    \caption{\textit{Esquemático de los conectores de UART y pines GND de verificación. Elaboración propia.}}
    \label{fig:sch_uart_gnd}
\end{figure}

Adicionalmente, se han incorporado jumpers para desacoplar y depurar ciertas conexiones críticas, como la batería y la entrada fotovoltaica hacia el AEM10941, así como las líneas \texttt{STATUS} del propio PMIC. La finalidad de estos jumpers es poder aislar el AEM10941 en caso de fallo y, si fuese necesario, conectar de forma externa una placa de evaluación del mismo componente para mantener la funcionalidad equivalente durante pruebas

\begin{figure}[H]
    \centering
    \includegraphics[width=0.8\linewidth]{imgs/sch/sch_others.png}
    \caption{\textit{Esquemático de los jumpers de desacoplo del PMIC. Elaboración propia.}}
    \label{fig:sch_others}
\end{figure}

\section{Diseño PCB}
Una vez finalizados los esquemáticos se pasó al desarrollo del \textit{layout} de la PCB. Para esta etapa se tuvieron muy en cuenta las referencias de la placa NUCLEO-WB55RG, especialmente en lo relativo a la colocación del microcontrolador, el subsistema RF y las recomendaciones generales de desacoplo. \\

La disposición de componentes se realizó agrupando, en la medida de lo posible, los elementos de cada subsistema, lo que facilitó el ruteado y redujo cruces innecesarios. Siempre que existían \textit{layouts} de referencia de los fabricantes, se tomaron como guía para minimizar riesgos de integración.

\subsection{Requerimientos del fabricante}
En cuanto a la fabricación, se seleccionó \textbf{JLCPCB} como proveedor. Por ello, antes de fijar el conjunto final de reglas de diseño, se revisaron sus capacidades de proceso para asegurar que el diseño fuese fabricable sin modificaciones. Para ello se consultaron las especificaciones del fabricante y se definieron reglas de diseño (DRC) acordes a los mínimos soportados. En la Figura \ref{fig:pcb_requirements} se muestran los valores mínimos de fabricación.

\begin{figure}[H]
    \centering
    \includegraphics[width=0.45\linewidth]{imgs/pcb_requirements.png}
    \caption{\textit{Requisitos mínimos de fabricación de JLCPCB.}}
    \label{fig:pcb_requirements}
\end{figure}

A partir de estos límites, se establecieron las reglas internas de ruteado utilizadas durante el diseño: un ancho por defecto de 0.20 mm para señales, vías de 0.60 mm de diámetro con taladro de 0.30 mm, y un ancho mayor en pistas de potencia, incrementado hasta 0.80 mm en los tramos donde el espacio lo permitía, con el objetivo de reducir la caída de tensión y mejorar la distribución de corriente.

\subsection{Apilado fisico (\textit{stack-up})}
Para escoger el número de capas y el apilado físico se tuvieron en cuenta varios factores. El primero fue el coste: las placas de 2 capas son las más económicas, seguidas por las de 4 capas. Sin embargo, dado el número de componentes y señales del sistema, se optó por un diseño de 4 capas, que permite un ruteado más limpio y el uso de planos de masa continuos. \\

Además, el \textit{stack-up} de 4 capas es especialmente ventajoso para el control de impedancia, necesario en el trazado RF y en la conexión hacia la antena. Este aspecto se desarrolla con más detalle en la Sección \ref{seccion:impedance_ctrl}.

\begin{figure}[H]
    \centering
    \includegraphics[width=1\linewidth]{imgs/pcb_4_layers.png}
    \caption{\textit{Grosores de la PCB escogidos. JLCPCB.}}
    \label{fig:pcb_4_layers}
\end{figure}

JLCPCB ofrece distintos espesores y opciones de fabricación para placas de 4 capas. Se escogió la opción estándar de 1.6 mm y 1 oz de cobre en las capas exteriores, tanto por disponibilidad como por coherencia con el control de impedancia. Dentro de los \textit{stack-ups} ofertados se seleccionó el \textbf{JLC04161H-7628}, mostrado en la Figura \ref{fig:pcb_stackup}, que define espesores dieléctricos y materiales y permite calcular anchos de pista para impedancias objetivo.

\begin{figure}[H]
    \centering
    \includegraphics[width=1\linewidth]{imgs/pcb_stackup.png}
    \caption{\textit{Stack-up JLC04161H-7628 seleccionado junto con espesores y materiales de cada capa. JLCPCB.}}
    \label{fig:pcb_stackup}
\end{figure}

Respecto a la distribución por capas, la capa \textit{top} se reservó para la mayor parte de componentes y pistas de señal, con el objetivo de facilitar el ensamblaje. La \textit{inner1} se dedicó a un plano continuo de GND, el cual mejora el retorno de corriente y ayuda al control de impedancia en líneas críticas. La \textit{inner2} se destinó principalmente a planos de alimentación, con la excepción de la zona RF, donde se mantuvo un plano de GND bajo la línea para asegurar una referencia adecuada. Por último, la capa \textit{bottom} se utilizó para las señales restantes cuando no era posible rutearlas en la \textit{top} debido a cortes o congestión de pistas, y para colocar el \textit{holder} de la pila de \textit{back-up}. La Tabla \ref{tabla:layers} muestra un resumen de la disposición de capas.

\begin{table}[H]
    \centering
    \begin{tabular}{|c|c|}
        \hline
        \rowcolor{LightSteelBlue1}
        \textbf{Capa} & \textbf{Contenido} \\
        \hline
        \textit{Top} & Componentes y señales \\
        \hline
        \textit{Inner1} & Plano GND \\
        \hline
        \textit{Inner2} & Islas de alimentación \\
        \hline
        \textit{Bottom} & Señales \\
        \hline
    \end{tabular}
    \caption{\textit{Resumen del contenido de cada capa. Elaboración propia.}}
    \label{tabla:layers}
\end{table}

\subsection{Condensadores de desacoplo}
Se prestó especial atención a la colocación y conexión de los condensadores de desacoplo, ya que tienen un papel fundamental en la estabilidad de los integrados y en la reducción de ruido en los raíles de alimentación. Para ello se siguieron las recomendaciones de diseño de la nota de aplicación \textit{SPMA056} \cite{spma056} que se observan en al Figura \ref{fig:decoupling_cap}.

\begin{figure}[H]
    \centering
    \includegraphics[width=0.85\linewidth]{imgs/decoupling_cap.png}
    \caption{\textit{Recomendaciones de colocación y conexión de condensadores de desacoplo. Texas Instruments.}}
    \label{fig:decoupling_cap}
\end{figure}

\subsection{Islas para los cristales}
Otra precaución adoptada para evitar problemas de arranque o inestabilidad fue aislar los osciladores en “islas” dedicadas. Estas islas consisten en una zona local con plano de GND cuidadosamente delimitado, rodeado por un anillo de vías y con un único punto de unión al plano de masa principal. \\

El objetivo es evitar que corrientes de retorno de señales digitales o de potencia circulen por debajo de las pistas del cristal, reduciendo el acoplo y mejorando la integridad del oscilador. En la Figura \ref{fig:pcb_osc} se muestran las islas implementadas para el HSE y el LSE.

\begin{figure}[H]
    \centering
    \begin{minipage}{0.60\textwidth}
        \centering
        \includegraphics[width=0.85\linewidth]{imgs/pcb/pcb_hse.png}
    \end{minipage}%
    \begin{minipage}{0.40\textwidth}
        \centering
        \includegraphics[width=0.85\linewidth]{imgs/pcb/pcb_lse.png}
    \end{minipage}
    \caption{\textit{Implementación de islas de masa y anillo de vías para los osciladores HSE y LSE. Elaboración propia.}}
    \label{fig:pcb_osc}
\end{figure}

\subsection{RF} \label{seccion:impedance_ctrl}
La última sección crítica del \textit{layout} fue el subsistema RF, tanto la conexión como la geometria de la antena. \\

Para dimensionar las pistas de impedancia controlada se utilizó la herramienta de JLCPCB para cálculo de impedancia en función del \textit{stack-up} seleccionado \cite{impedance_calc}. Se optó por una disposición de tipo coplanar con plano de masa, en la que la pista RF discurre encima de una referencia de GND y además está rodeada por cobre de GND lateral en la misma capa (Figura \ref{fig:coplanar}). Esta configuración mejora el control de impedancia y confina mejor el campo electromagnético.

\begin{figure}[H]
    \centering
    \begin{minipage}{0.70\textwidth}
        \centering
        \includegraphics[width=0.85\linewidth]{imgs/coplanar.png}
    \end{minipage}%
    \begin{minipage}{0.30\textwidth}
        \centering
        \includegraphics[width=0.85\linewidth]{imgs/pcb/pcb_coplanar.png}
    \end{minipage}
    \caption{\textit{Estructura de línea coplanar con referencia de GND y su implementación en la PCB para control de impedancia.}}
    \label{fig:coplanar}
\end{figure}

Para el cálculo se requieren el \textit{stack-up}, la impedancia objetivo y la separación entre la pista y el plano GND coplanar (0.16 mm en este diseño). Introduciendo estos parámetros en la calculadora, se obtiene el ancho óptimo de pista. Para simplificar el diseño y ajustarlo a valores redondeados, se adoptaron los siguientes anchos finales: para 50 $\Omega$ la calculadora devuelve 0.2695 mm y se utilizó 0.28 mm (Figura \ref{fig:pcb_50R}); para 62 $\Omega$ la calculadora devuelve 0.1537 mm y se utilizó 0.16 mm (Figura \ref{fig:pcb_62R}).

\begin{figure}[H]
    \centering
    \includegraphics[width=1\linewidth]{imgs/pcb_50R.png}
    \caption{\textit{Cálculo del ancho de pista para 50 $\Omega$ en el stack-up seleccionad.}}
    \label{fig:pcb_50R}
\end{figure}

\begin{figure}[H]
    \centering
    \includegraphics[width=1\linewidth]{imgs/pcb_62R.png}
    \caption{\textit{Cálculo del ancho de pista para 62 $\Omega$ en el stack-up seleccionad.}}
    \label{fig:pcb_62R}
\end{figure}

Para el diseño de la antena en PCB se siguieron las recomendaciones de STMicroelectronics recogidas en la \textit{Application Note AN5129} \cite{an5129}. En la Figura \ref{fig:anthenna} aparecen las dimensiones recomendadas.

\begin{figure}[H]
    \centering
    \includegraphics[width=0.8\linewidth]{imgs/anthenna.png}
    \caption{\textit{Dimensiones recomendadas para la antena PCB. STMicroelectronics.}}
    \label{fig:anthenna}
\end{figure}

\subsection{Resultado final}

\begin{figure}[H]
    \centering
    \includegraphics[width=1\linewidth]{imgs/pcb/pcb_top.png}
    \caption{\textit{Capa superior del layout de la PCB final (componentes y señales). Elaboración propia.}}
    \label{fig:pcb_top}
\end{figure}

\begin{figure}[H]
    \centering
    \includegraphics[width=1\linewidth]{imgs/pcb/pcb_inner1.png}
    \caption{\textit{Primera capa interior del layout de la PCB final (plano GND). Elaboración propia.}}
    \label{fig:pcb_inner1}
\end{figure}

\begin{figure}[H]
    \centering
    \includegraphics[width=1\linewidth]{imgs/pcb/pcb_inner2.png}
    \caption{\textit{Segunda capa interior del layout de la PCB final (planos de alimentación). Elaboración propia.}}
    \label{fig:pcb_inner2}
\end{figure}

\begin{figure}[H]
    \centering
    \includegraphics[width=1\linewidth]{imgs/pcb/pcb_bottom.png}
    \caption{\textit{Capa inferior del layout de la PCB final (señales restantes y pila CR2032). Elaboración propia.}}
    \label{fig:pcb_bottom}
\end{figure}

\begin{figure}[H]
    \centering
    \includegraphics[width=1\linewidth]{imgs/pcb/pcb_front.png}
    \caption{\textit{Vista frontal del ensamblaje de la PCB. Elaboración propia.}}
    \label{fig:pcb_front}
\end{figure}

\begin{figure}[H]
    \centering
    \includegraphics[width=1\linewidth]{imgs/pcb/pcb_back.png}
    \caption{\textit{Vista posterior del ensamblaje de la PCB. Elaboración propia.}}
    \label{fig:pcb_back}
\end{figure}

\begin{figure}[H]
    \centering
    \includegraphics[width=1\linewidth]{imgs/pcb/pcb_3d.png}
    \caption{\textit{Render 3D del ensamblaje final de la PCB. Elaboración propia.}}
    \label{fig:pcb_3d}
\end{figure}

\section{Fabricación y montaje}
Cuando se finalizó el diseño de la PCB se generaron los archivos de fabricación necesarios para realizar el pedido a JLCPCB. En concreto, los ficheros principales fueron los Gerbers, la lista de materiales (BOM) y el archivo de posición (\textit{Pick \& Place}), que permiten tanto la fabricación de la placa como el ensamblaje automático de los componentes.

\subsection{Proceso de elección de componentes de JLCPCB}
JLCPCB ofrece dos modalidades de fabricación: fabricar únicamente la PCB para que el ensamblaje lo realice el usuario, o fabricar la PCB y realizar también el ensamblaje con los componentes seleccionados. Dado que el sistema incorpora encapsulados QFN, difíciles de soldar manualmente con herramientas básicas, y pasivos de tamaño 0402, se optó por la segunda opción, delegando el ensamblaje automático al fabricante para reducir riesgo de fallo por soldadura. \\

Para que JLCPCB pueda montar los componentes es necesario asignar a cada referencia un número de parte de su catálogo \cite{jlcpcb_shop}. Durante el proceso de selección se tuvo en cuenta si los componentes pertenecían a la categoría \textit{Basic} o \textit{Extended}, ya que estos últimos conllevan un coste adicional por componente, lo que puede penalizar el presupuesto si se utilizan muchos componentes en dicha categoría. \\

En general no hubo problemas para encontrar equivalentes adecuados en el catálogo de JLCPCB. La excepción fue el AEM10941, que en el momento del pedido no estaba disponible para ensamblaje. Por este motivo se decidió comprar dicho integrado por separado y soldarlo manualmente en la placa tras la recepción, manteniendo el resto del montaje realizado por el fabricante.

\subsection{BOM} \label{seccion:bom}
En la Tabla \ref{tabla:bom} se proporciona un resumen de la BOM resultante del proyecto.

\scriptsize
\begin{longtable}{|m{0.2\linewidth}|l|l|m{0.28\linewidth}|l|}
    \hline
    \rowcolor{LightSteelBlue1}
    \textbf{Reference} & \textbf{Value} & \textbf{Component} & \textbf{Description} & \textbf{JLCPCB Part \#} \\
    \hline

    C7 & 22uF & CL10A226MQ8NRNC & 22uF 6.3V X5R ±20\% 0603 Multilayer Ceramic Capacitors SMD & C59461 \\ 
    \hline
    C9 & 150uF & GRM31CR60J157ME11L & 150uF 6.3V X5R ±20\% 1206 Multilayer Ceramic Capacitors SMD & C528968 \\ 
    \hline
    C20 & 100pF & 0402CG101J500NT & 100pF 50V C0G ±5\% 0402 Multilayer Ceramic Capacitors SMD & C1546 \\ 
    \hline
    C30 & 1.2pF & GJM1555C1H1R2WB01D & 1.2pF 50V C0G 0402 Multilayer Ceramic Capacitor SMD & C76899 \\ 
    \hline
    C1, C2, C3, C4, C10, C11, C12, C13, C14, C16, C17, C18, C19, C21, C23, C26, C27, C28 & 100nF & CL05B104KO5NNNC & 100nF 16V X7R ±10\% 0402 Multilayer Ceramic Capacitors SMD & C1525 \\ 
    \hline
    C15, C22 & 4.7uF & CL05A475MP5NRNC & 4.7uF 10V X5R ±20\% 0402 Multilayer Ceramic Capacitors SMD & C23733 \\ 
    \hline
    C24, C25 & 12pF & 0402CG120J500NT & 12pF 50V C0G ±5\% 0402 Multilayer Ceramic Capacitor SMD & C1547 \\ 
    \hline
    C5, C6, C8 & 10uF & CL10A106KP8NNNC & 10uF 10V X5R ±10\% 0603 Multilayer Ceramic Capacitors SMD & C19702 \\ 
    \hline
    D1 & D\_Schottky & 1N5819WS & 40V 1A Schottky Diode SOD-323 & C191023 \\ 
    \hline
    FLT1 & LPF\_2.4GHz & MLPF-WB55-01E3 & 2.4GHz\textasciitilde2.5GHz 50$\Omega$ 0.9dB Low Pass RF Filter SMD-6P,1.6x1mm & C2651048 \\ 
    \hline
    L1 & 10uH & LPS4018-103MRC & 10uH 200m$\Omega$ 1.3A 40MHz ±20\% Magnetic Shielded Inductor 3.9x3.9mm SMD & C6127589 \\ 
    \hline
    L2 & 10uH & MLZ1608M100WT000 & 10uH 1.05$\Omega$ 90mA ±20\% Inductor 0603 SMD & C76798 \\ 
    \hline
    L3 & 600@100MHz & GZ1608D601TF & 600$\Omega$@100MHz 450m$\Omega$ 200mA Ferrite Beads ±25\% 0603 SMD & C1002 \\ 
    \hline
    L4 & 10nH & SDCL1005C10NJTDF & 10nH 400m$\Omega$ 300mA Inductor ±5\% 0402 SMD & C27147 \\ 
    \hline
    L5 & 10uH & SDFL2012S100KTF & 10uH 1.15$\Omega$ 15mA Inductor ±10\% 0805 SMD & C1046 \\ 
    \hline
    L6 & 3.6nH & LQG15HS3N6S02D & 3.6nH 0.14$\Omega$ 750mA Inductor ±5\% 0402 SMD & C19216 \\ 
    \hline
    LED1 & Green & KT-0805G & 2.6-3.1V 5mA 430mcd 525nm 100mW Emerald Green LED Water Clear 0805 SMD & C2297 \\ 
    \hline
    LED2 & White & KT-0805W & 2.6-3.2V 25mA 350mcd 80mW White LED Difuse Lens 0805 SMD & C34499 \\ 
    \hline
    Q1 & AO3401A & AO3401A & P-Channel MOSFET SOT-23 & C15127 \\ 
    \hline
    R16 & 470 & 0603WAF4700T5E & 470$\Omega$ 75V 100mW Thick Film Resistor ±1\% ±100ppm/℃ 0603 SMD & C23179 \\ 
    \hline
    R17 & 300 & 0603WAF3000T5E & 300$\Omega$ 75V 100mW Thick Film Resistor ±1\% ±100ppm/℃ 0603 SMD & C23025 \\ 
    \hline
    R18 & 100k & 0603WAF1003T5E & 100k$\Omega$ 75V 100mW Thick Film Resistor ±1\% ±100ppm/℃ 0603 SMD & C25803 \\ 
    \hline
    R1, R2, R3 & 0.22 & RLP25FEER220 & 220m$\Omega$ 2W Current Sense Resistor ±1\% ±50ppm/℃ 2512 SMD & C912987 \\ 
    \hline
    R10, R11, R12, R13, R14, R15, R19 & 10k & 0402WGF1002TCE & 10k$\Omega$ 50V 62.5mW Thick Film Resistor ±1\% ±100ppm/℃ 0402 SMD & C25744 \\ 
    \hline
    R4, R5, R6, R7, R8, R9 & 10 & 0402WGF100JTCE & 10$\Omega$ 50V 62.5mW Thick Film Resistor ±1\% ±200ppm/℃ 0402 SMD & C25077 \\ 
    \hline
    SW1 & SW\_Reset & TS-1187A-B-A-B & 100.000 cycles 12V Gold Tactile Switch 5.1x5.1mm SMD-4P & C318884 \\ 
    \hline
    U1 & STM32WB55RG & STM32WB55RGV6 & STMicroelectronics Arm Cortex-M4 MCU, 64 MHz, 1.71-3.6V, 49 GPIO, VFQFPN68 & C401505 \\ 
    \hline
    U2 & AEM10941 & AEM10941-QFN & Highly Efficient Dual-Output PMIC, QFN-28 & C9900027046 \\ 
    \hline
    U3 & INA3221 & INA3221AIRGVR & Triple-Channel High-Side Shunt and Bus Voltage Monitor, I2C \& SMBUS, QFN-16 & C181255 \\ 
    \hline
    U4 & MB85RS256B & MB85RS256BPNF & 256Kbit 2.7-3.6V 5mA FRAM memory SPI SOIC-8 & C8742 \\ 
    \hline
    X1 & 32MHz & Q22FA12800025 & 8pF 32MHz 60$\Omega$ SMD2016-4P Crystal & C187794 \\ 
    \hline
    X2 & 32.768kHz & NX2012SA-32.768KHZ & 9pF 32.768kHz 80$\Omega$ SMD2012-2P Crystal & C891748 \\ 
    \hline
\caption{\textit{Lista de materiales (BOM) del diseño principal, incluyendo referencias, valores y número de parte de JLCPCB. Elaboración propia.}}
\label{tabla:bom}
\end{longtable}
\normalsize

\subsection{Proceso de ensamblaje} \label{seccion:ensamblaje}
Una vez enviados los archivos de fabricación y ensamblaje, se recibieron las placas tras el plazo de entrega correspondiente. Al recibirlas, se realizó una verificación incremental del funcionamiento por subsistemas, con el objetivo de detectar posibles fallos de montaje o de diseño de manera controlada. \\

En primer lugar se comprobó el encendido del LED de POWER, verificando la correcta alimentación y la presencia del bus de 3.3 V. A continuación se validó la MCU, ya que era uno de los puntos con mayor incertidumbre en la integración: se comprobó la programación mediante SWD y la ejecución básica de un programa. Posteriormente se verificó el bloque de monitorización de potencia, la memoria FRAM y la adquisición de datos de los sensores. Finalmente se comprobó el subsistema con mayor sensibilidad al \textit{layout}: la parte de RF, verificando la comunicación BLE y confirmando que el trazado y la antena funcionaban conforme a lo esperado. \\

Tras confirmar que el ensamblaje era correcto, se procedió al montaje manual de los elementos no ensamblados por el fabricante. En particular, se soldó el AEM10941 junto con conectores y \textit{jumpers}. La soldadura del AEM10941 resultó especialmente exigente debido a su encapsulado QFN, que requiere control térmico y correcta cantidad de estaño en los pads. Dado que el resto de componentes ya estaban montados, no era viable un proceso de \textit{reflow} completo en un horno, por lo que se recurrió a una pistola de aire caliente para su soldadura. Tras varias iteraciones de práctica y ajustes de temperatura y flujo, se consiguió una soldadura correcta del integrado y del resto de pines auxiliares.

\begin{figure}[H]
    \centering
    \includegraphics[width=1\linewidth]{imgs/photo/circuit_6.jpg}
    \caption{\textit{PCB final completamente soldada. Elaboración propia.}}
    \label{fig:circuit_6}
\end{figure}

\section{Integración en la lanza}
Para definir con claridad la integración física de la PCB dentro de la lanza meteorológica, se realizó un diseño 3D del conjunto completo. Las Figuras \ref{fig:full_spear}, \ref{fig:pcb_case} y \ref{fig:spear_sensors} muestran imágenes del diseño.


\begin{figure}[H]
    \centering
    \begin{minipage}{0.2\textwidth}
        \centering
        \includegraphics[width=0.9\linewidth]{imgs/3d/full_spear.png}
    \end{minipage}%
    \begin{minipage}{0.22\textwidth}
        \centering
        \includegraphics[width=0.9\linewidth]{imgs/3d/full_spear_45.png}
    \end{minipage}%
    \begin{minipage}{0.58\textwidth}
        \centering
        \includegraphics[width=0.9\linewidth]{imgs/3d/solar_panels_45.png}
    \end{minipage}
    \caption{\textit{Diseño 3D de la lanza, los paneles solares y el sensor de irradiancia. Elaboración propia.}}
    \label{fig:full_spear}
\end{figure}

\begin{figure}[H]
    \centering
    \begin{minipage}{0.5\textwidth}
        \centering
        \includegraphics[width=1\linewidth]{imgs/3d/pcb_case.png}
    \end{minipage}%
    \begin{minipage}{0.5\textwidth}
        \centering
        \includegraphics[width=1\linewidth]{imgs/3d/pcb_case_45.png}
    \end{minipage}
    \caption{\textit{Diseño 3D del encapsulado de la PCB diseñada. Elaboración propia.}}
    \label{fig:pcb_case}
\end{figure}

\begin{figure}[H]
    \centering
    \begin{minipage}{0.22\textwidth}
        \centering
        \includegraphics[width=0.9\linewidth]{imgs/3d/air_sensor_spear_45.png}
    \end{minipage}%
    \begin{minipage}{0.56\textwidth}
        \centering
        \includegraphics[width=0.9\linewidth]{imgs/3d/air_sensor_45.png}
    \end{minipage}%
    \begin{minipage}{0.22\textwidth}
        \centering
        \includegraphics[width=0.9\linewidth]{imgs/3d/spear_end_45.png}
    \end{minipage}
    \caption{\textit{Diseño 3D del encapsulado y agujeros de los diferentes sensores. Elaboración propia.}}
    \label{fig:spear_sensors}
\end{figure}

\chapter{Desarrollo software}

\section{Entorno y herramientas}
Para el desarrollo de \textit{firmware} en microcontroladores STM32 existen distintas alternativas, sin embargo, para este proyecto se ha optado por el ecosistema oficial de STMicroelectronics por ser la opción más integrada y robusta: facilita la configuración del microcontrolador, genera automáticamente el código base de inicialización y ofrece herramientas consolidadas para programación y depuración. \\

Las herramientas principales empleadas han sido:

\begin{itemize}
    \item \textbf{STM32CubeMX:} herramienta de configuración y generación de proyecto. Permite seleccionar el microcontrolador, habilitar periféricos (I2C, SPI, ADC, RTC, GPIO...), asignar pines, configurar relojes, opciones de bajo consumo y generar el código de inicialización y la estructura del proyecto.
    \item \textbf{STM32CubeIDE:} entorno de desarrollo integrado utilizado para editar, compilar y depurar el \textit{firmware}. Integra compilador y depurador, y permite programar y depurar la MCU mediante ST-LINK.
    \item \textbf{STM32CubeProgrammer:} herramienta para programar y verificar el microcontrolador. Permite cargar binarios en memoria Flash, comprobar su integridad, leer/escribir memoria y configurar parámetros del dispositivo cuando es necesario.
\end{itemize}

En el caso del STM32WB55, para poder utilizar la pila BLE fue necesario cargar previamente el \textit{stack} proporcionado por el fabricante en la memoria Flash. Para ello se empleó STM32CubeProgrammer y, siguiendo las indicaciones de STMicroelectronics, se programó en la dirección \texttt{0x080CE000} el archivo «\texttt{stm32wb5x\_BLE\_Stack\_full\_fw.bin}» \cite{ble_stack_github}.

\section{Arquitectura del \textit{firmware}}
\begin{figure}[H]
    \centering
    \includegraphics[width=1\linewidth]{imgs/firmware_arch.png}
    \caption{\textit{Diagrama conceptual de la arquitectura del firmware del sistema. Elaboración propia.}}
    \label{fig:firmware_arch}
\end{figure}

La Figura \ref{fig:firmware_arch} muestra los flujos principales del \textit{firmware}. El ciclo normal se desencadena por el RTC, realizando la lectura de sensores y registro en la FRAM. Los eventos EXTI, BLE y UART se atienden de forma específica y, una vez completada la tarea, el sistema vuelve a un modo de bajo consumo para maximizar la autonomía.

\section{\textit{Drivers}}
Todos los \textit{drivers} de los componentes fueron desarrollados específicamente para este proyecto. Se decidió implementar \textit{drivers} propios en lugar de utilizar librerías de terceros para mantener el máximo control sobre el acceso al \textit{hardware}, reducir dependencias externas y comprender en profundidad el funcionamiento real de cada componente. \\

La programación de los \textit{drivers} ha seguido estos principios:

\begin{itemize}
    \item \textbf{API sencilla y consistente:} se ha buscado que todos los \textit{drivers} tengan una interfaz similar, de manera que su integración en el firmware sea directa y homogénea.
    \item \textbf{Funcionalidades completas pero solo las necesarias:} se han implementado las operaciones requeridas por el sistema, evitando incluir características poco utilizadas que aumenten la complejidad, el tamaño de código y la superficie de fallo. Aun así, el diseño se ha dejado abierto para poder ampliar funcionalidades si en el futuro fuese necesario.
    \item \textbf{Portabilidad:} los \textit{drivers} no acceden a registros del microcontrolador directamente, sino que se apoyan en la HAL de ST para las comunicaciones. Esto desacopla el código del microcontrolador concreto y facilita reutilizarlo en otros proyectos con cambios mínimos.
\end{itemize}

Además, la estructura de los \textit{drivers} sigue un patrón común. En general, cada \textit{driver} se compone de:

\begin{itemize}
    \item Un \textit{struct} que almacena la información imprescindible para su funcionamiento.
    \item Un conjunto de \textit{defines} y enumeraciones para facilitar el uso del componente (direcciones de registros, máscaras de bits, constantes de conversión, selección de modos...).
    \item Un conjunto de funciones públicas que exponen las operaciones necesarias para el sistema (inicialización, lectura/escritura...).
\end{itemize}

\subsection{INA3221}
El \textit{driver} del INA3221 se comunica mediante el bus I2C. El integrado trabaja como esclavo y su dirección I2C por defecto es \texttt{0x40}. Su funcionamiento se basa en la escritura y lectura de registros internos: por un lado registros de configuración y por otro registros de medida (tensión de bus y tensión en \textit{shunt}) para cada canal. \\

El \textit{struct} del componente se denomina \texttt{INA3221\_t} y contiene el \textit{handle} de I2C de la librería HAL, los valores de las resistencias \textit{shunt} y una serie de parámetros de configuración necesarios para la inicialización: modo de promediado, tiempos de conversión de bus y \textit{shunt}, y modo de operación. \\

Las funciones públicas expuestas por el \textit{driver} son:
\begin{itemize}
    \item \texttt{HAL\_StatusTypeDef INA3221\_Init(INA3221\_t *dev)}: escribe el registro de configuración del INA3221 con los parámetros almacenados en el \textit{struct}.
    \item \texttt{HAL\_StatusTypeDef INA3221\_ReadVoltage(INA3221\_t *dev, uint8\_t channel, float *busVoltage, float *shuntVoltage)}: lee los registros de tensión de bus y tensión en \textit{shunt} del canal seleccionado y devuelve ambas magnitudes ya convertidas a voltios.
    \item \texttt{float INA3221\_CalculateCurrent\_mA(INA3221\_t *dev, uint8\_t channel, float shuntVoltage)}: calcula la corriente del canal a partir de la caída en la resistencia \textit{shunt} y el valor de \textit{shunt} configurado.
    \item \texttt{float INA3221\_CalculatePower\_mW(float busVoltage, float shuntCurrent\_mA)}: calcula la potencia aproximada a partir de tensión de bus y corriente.
\end{itemize}

Internamente, la inicialización (\texttt{INA3221\_Init}) construye el valor del registro de configuración combinando los campos de promediado, tiempos de conversión y modo de operación, y lo escribe mediante una operación I2C del tipo \textit{memory write}. La lectura sigue una secuencia simple: valida el canal solicitado (1 a 3), calcula la dirección de los registros a leer en función del canal, realiza dos lecturas I2C de 16 bits (registro de \textit{shunt} y registro de bus) y recompone los bytes leídos y aplica las conversiones a unidades físicas. En la conversión se tiene en cuenta el formato del dato de medida (bits no significativos) y el signo en la lectura de \textit{shunt}, ya que la caída puede ser positiva o negativa. \\

El proceso de uso habitual del componente sería:
\begin{enumerate}
    \item Asignar los valores de configuración del \textit{struct} \texttt{INA3221\_t}.
    \item Llamar a \texttt{INA3221\_Init} para programar el registro de configuración.
    \item Llamar a \texttt{INA3221\_ReadVoltage} para cada canal y obtener \texttt{busVoltage} y \texttt{shuntVoltage}.
    \item Calcular la corriente con \texttt{INA3221\_CalculateCurrent\_mA} usando la tensión de \textit{shunt} medida.
    \item Calcular la potencia con \texttt{INA3221\_CalculatePower\_mW} a partir de tensión de bus y corriente.
\end{enumerate}

\subsection{MB85RS256B}
El \textit{driver} de la memoria FRAM MB85RS256B se comunica mediante el bus SPI. El componente funciona como esclavo y su uso se basa en el envío de un comando junto con una dirección de memoria y, según la operación, los datos a transmitir o a recibir. Al tratarse de FRAM, la escritura es directa (no es necesario gestionar borrado por páginas), lo que simplifica el almacenamiento periódico de registros. \\

El \textit{struct} del componente se denomina \texttt{MB85RS256B\_t} y contiene el \textit{handle} del periférico SPI de la librería HAL, además de la información necesaria para controlar el pin \texttt{CS}. \\

Las funciones públicas expuestas por el \textit{driver} son:
\begin{itemize}
    \item \texttt{HAL\_StatusTypeDef MB85RS256B\_Init(MB85RS256B\_t *dev)}: inicializa el \textit{driver}.
    \item \texttt{HAL\_StatusTypeDef MB85RS256B\_Read(MB85RS256B\_t *dev, uint16\_t addr, uint8\_t *data, size\_t len)}: lee \texttt{len} bytes a partir de la dirección \texttt{addr} y los almacena en \texttt{data}.
    \item \texttt{HAL\_StatusTypeDef MB85RS256B\_Write(MB85RS256B\_t *dev, uint16\_t addr, const uint8\_t *data, size\_t len)}: escribe \texttt{len} bytes desde \texttt{data} a partir de la dirección \texttt{addr}.
\end{itemize}

Internamente, todas las transacciones delimitan el bus mediante el pin \texttt{CS}: se baja \texttt{CS} al inicio y se sube al finalizar. Para una lectura, la función \texttt{MB85RS256B\_Read} construye una cabecera de 3 bytes con el comando \texttt{READ} y la dirección (MSB primero), transmite dicha cabecera y a continuación recibe \texttt{len} bytes por SPI. Para una escritura, \texttt{MB85RS256B\_Write} habilita previamente la escritura (comando \texttt{WREN} implementado como función interna), transmite la cabecera (\texttt{WRITE} + dirección) y después transmite el bloque de datos. Al finalizar la operación, la escritura se deshabilita de nuevo (comando \texttt{WRDI}) para evitar escrituras accidentales. \\

El proceso de uso habitual del componente sería:
\begin{enumerate}
    \item Asignar los valores de configuración del \textit{struct} \texttt{MB85RS256B\_t}.
    \item Llamar a \texttt{MB85RS256B\_Init} para inicializar el \textit{driver}.
    \item Llamar a \texttt{MB85RS256B\_Write} para almacenar un bloque de datos en una dirección determinada.
    \item Llamar a \texttt{MB85RS256B\_Read} para recuperar el bloque almacenado.
\end{enumerate}


\subsection{TSL2591}
El \textit{driver} del TSL2591 se comunica mediante el bus I2C. El integrado trabaja como esclavo y su dirección I2C por defecto es \texttt{0x29}. Su funcionamiento se basa en la escritura y lectura de registros internos: por un lado registros de control y por otro registros de medida que se leen para obtener el nivel de iluminación. \\

El \textit{struct} del componente se denomina \texttt{TSL2591\_t} y contiene el \textit{handle} de I2C de la librería HAL, junto con los parámetros de configuración principales del sensor: ganancia y tiempo de integración. Estos parámetros determinan el rango dinámico y la sensibilidad del sensor, y se emplean durante la inicialización y el cálculo posterior de \textit{lux}. \\

Las funciones públicas expuestas por el \textit{driver} son:
\begin{itemize}
    \item \texttt{HAL\_StatusTypeDef TSL2591\_Init(TSL2591\_t *dev)}: inicializa el sensor y configura el registro de control.
    \item \texttt{HAL\_StatusTypeDef TSL2591\_ReadChannels(TSL2591\_t *dev, uint16\_t *ch0, uint16\_t *ch1)}: lee los registros de datos y devuelve los dos canales del sensor: canal completo y canal infrarrojo.
    \item \texttt{float TSL2591\_CalculateLux(TSL2591\_t *dev, uint16\_t full, uint16\_t ir)}: calcula el valor de iluminancia en \textit{lux} a partir de las lecturas de los canales, teniendo en cuenta la ganancia y el tiempo de integración configurados.
    \item \texttt{float TSL2591\_CalculateIrradiance(float lux)}: obtiene una irradiancia aproximada a partir de lux utilizando un factor de eficacia luminosa definido en el \textit{driver}.
\end{itemize}

Internamente, el \textit{driver} accede a los registros del sensor utilizando un \textit{command bit} que se añade a la dirección del registro. En la inicialización, primero se habilita el sensor (registro \textit{ENABLE}) y después se configura el registro \textit{CONTROL} combinando los campos de integración y ganancia en un único byte de configuración. La lectura de canales realiza una lectura I2C de 4 bytes consecutivos a partir del registro de datos del canal 0, y reconstruye los dos valores de 16 bits correspondientes al espectro (\texttt{CH0}) y infrarrojo (\texttt{CH1}). A partir de estas lecturas, el cálculo de \textit{lux} ajusta la medida según la configuración actual. \\

El proceso de uso habitual del componente sería:
\begin{enumerate}
    \item Asignar los valores de configuración del \textit{struct} \texttt{TSL2591\_t}.
    \item Llamar a \texttt{TSL2591\_Init} para habilitar el sensor y programar la ganancia e integración.
    \item Llamar a \texttt{TSL2591\_ReadChannels} para obtener \texttt{full} e \texttt{ir}.
    \item Calcular los \textit{lux} con \texttt{TSL2591\_CalculateLux} a partir de \texttt{full} e \texttt{ir}.
    \item Convertir \textit{lux} a irradiancia aproximada con \texttt{TSL2591\_CalculateIrradiance}.
\end{enumerate}

\subsection{SHT3x}
El \textit{driver} del SHT3x se comunica mediante el bus I2C. El integrado trabaja como esclavo y su dirección I2C por defecto es \texttt{0x44}. Su funcionamiento se basa en el envío de \textit{comandos} y la lectura de datos a través del bus: por un lado comandos de control y por otro la lectura de las medidas de temperatura y humedad. \\

El \textit{struct} del componente se denomina \texttt{SHT3X\_t} y contiene el \textit{handle} de I2C de la librería HAL junto con los parámetros principales de configuración utilizados por el \textit{driver}: el modo de repetibilidad y el uso de \textit{clock stretching}. Estos parámetros determinan qué comando de medida se envía al sensor y el tiempo de conversión necesario antes de realizar la lectura. \\

Las funciones públicas expuestas por el \textit{driver} son:
\begin{itemize}
    \item \texttt{HAL\_StatusTypeDef SHT3X\_Init(SHT3X\_t *dev)}: inicializa el componente realizando un \textit{soft reset}.
    \item \texttt{HAL\_StatusTypeDef SHT3X\_ReadRaw(SHT3X\_t *dev, uint16\_t *rawT, uint16\_t *rawRH)}: realiza una medida \textit{single-shot} y devuelve los valores crudos de temperatura y humedad.
    \item \texttt{HAL\_StatusTypeDef SHT3X\_ReadSingleShot(SHT3X\_t *dev, float *temp\_c, float *rh\_perc)}: realiza una medida \textit{single-shot} y devuelve temperatura en $^\circ$C y humedad relativa en \%.
    \item \texttt{float SHT3X\_CalculateDewpoint(float temp\_c, float rh\_perc)}: calcula el punto de rocío a partir de temperatura y humedad.
    \item \texttt{HAL\_StatusTypeDef SHT3X\_ReadStatus(SHT3X\_t *dev, uint16\_t *status)}: lee el registro de estado del sensor.
    \item \texttt{HAL\_StatusTypeDef SHT3X\_ClearStatus(SHT3X\_t *dev)}: limpia el registro de estado.
    \item \texttt{HAL\_StatusTypeDef SHT3X\_Heater(SHT3X\_t *dev, SHT3X\_Heater\_t state)}: habilita o deshabilita el calefactor interno.
\end{itemize}

Internamente, para realizar una medida \textit{single-shot}, el \textit{driver} selecciona el comando apropiado en función de \texttt{repeatability} y \texttt{clockStretch}. A continuación envía dicho comando por I2C. Si se utiliza el modo sin \textit{clock stretching}, el \textit{driver} espera el tiempo de conversión correspondiente antes de leer los datos. La lectura devuelve 6 bytes (temperatura + CRC y humedad + CRC), y el \textit{driver} verifica la integridad mediante CRC antes de reconstruir los valores crudos. Finalmente, \texttt{SHT3X\_ReadSingleShot} convierte los valores a unidades físicas ($^\circ$C y \%) y limita la humedad relativa al rango [0,100]. \\

El proceso de uso habitual del componente sería:
\begin{enumerate}
    \item Asignar los valores de configuración del \textit{struct} \texttt{SHT3X\_t}.
    \item Llamar a \texttt{SHT3X\_Init} para reiniciar e inicializar el sensor.
    \item Llamar a \texttt{SHT3X\_ReadSingleShot} para obtener \texttt{temp\_c} y \texttt{rh\_perc}.
    \item Calcular punto de rocío con \texttt{SHT3X\_CalculateDewpoint}.
\end{enumerate}

\subsection{DS18B20}
El \textit{driver} del DS18B20 se comunica mediante el protocolo 1-Wire usando un único GPIO configurado como línea bidireccional. A diferencia de I2C/SPI, el 1-Wire requiere una temporización precisa, por lo que el \textit{driver} implementa la capa física mediante generación de pulsos, lectura/escritura de bits y retardos en microsegundos. \\

El \textit{struct} del componente se denomina \texttt{DS18B20\_t} y contiene el puerto y pin GPIO usados como bus 1-Wire, además de la resolución seleccionada, que determina principalmente el tiempo necesario para completar una conversión de temperatura. \\

Las funciones públicas expuestas por el \textit{driver} son:
\begin{itemize}
    \item \texttt{HAL\_StatusTypeDef DS18B20\_Init(DS18B20\_t *dev)}: inicializa el \textit{driver} y el sistema de retardos de alta resolución utilizado para el 1-Wire.
    \item \texttt{HAL\_StatusTypeDef DS18B20\_ReadTemperature(DS18B20\_t *dev, float *temp\_c)}: inicia una conversión, espera a que finalice según la resolución configurada y lee el \textit{scratchpad} para devolver la temperatura en $^\circ$C.
    \item \texttt{HAL\_StatusTypeDef DS18B20\_ReadROM(DS18B20\_t *dev, uint8\_t rom[8])}: lee el \textit{ROM code} de 64 bits del sensor (útil para identificación) y valida la lectura mediante CRC.
\end{itemize}

Internamente, el \textit{driver} implementa las primitivas 1-Wire necesarias: pulso de \textit{reset} y detección de presencia, escritura y lectura de bits y bytes, y cálculo de CRC8. La función \texttt{DS18B20\_ReadTemperature} se realiza en dos transacciones: primero envía \texttt{SKIP\_ROM} y el comando \texttt{CONVERT\_T} para iniciar la conversión; después espera un tiempo máximo que depende de la resolución y finalmente vuelve a hacer \textit{reset}, envía \texttt{READ\_SCRATCH} y lee 9 bytes. La integridad se comprueba con CRC y, si es correcto, se reconstruye el valor crudo y se convierte a $^\circ$C. \\

El proceso de uso habitual del componente sería:
\begin{enumerate}
    \item Asignar los valores de configuración del \textit{struct} \texttt{DS18B20\_t}.
    \item Llamar a \texttt{DS18B20\_Init} para inicializar el \textit{driver}.
    \item Llamar a \texttt{DS18B20\_ReadTemperature} para obtener la temperatura en $^\circ$C.
\end{enumerate}

\subsection{SEN0308}
El \textit{driver} del SEN0308 se basa en la lectura de una señal analógica proporcional a la humedad del suelo. A nivel de firmware, el sensor se lee mediante el ADC del microcontrolador, utilizando las funciones de la HAL para iniciar la conversión, esperar a que finalice y obtener el valor digital resultante. \\

El \textit{struct} del componente se denomina \texttt{SEN0308\_t} y contiene el \textit{handle} del ADC de la librería HAL, junto con parámetros de calibración para convertir la lectura a una escala relativa: \texttt{airRaw} (lectura en seco/aire) y \texttt{waterRaw} (lectura en agua/suelo saturado). Adicionalmente, el \textit{struct} incluye el puerto y pin asociados al sensor para posibles tareas de control. \\

Las funciones públicas expuestas por el \textit{driver} son:
\begin{itemize}
    \item \texttt{HAL\_StatusTypeDef SEN0308\_Init(SEN0308\_t *dev)}: inicializa el \textit{driver}.
    \item \texttt{HAL\_StatusTypeDef SEN0308\_ReadRaw(SEN0308\_t *dev, uint16\_t *rawMoisture)}: realiza una conversión ADC y devuelve la lectura cruda del sensor.
    \item \texttt{HAL\_StatusTypeDef SEN0308\_ReadRawAvg(SEN0308\_t *dev, uint16\_t *rawMoisture, uint8\_t numSamples)}: obtiene una lectura promediada para reducir ruido.
    \item \texttt{uint8\_t SEN0308\_CalculateRelative(SEN0308\_t *dev, uint16\_t rawMoisture)}: convierte la lectura cruda a un porcentaje relativo (0--100\%) usando los puntos de calibración definidos.
\end{itemize}

Internamente, \texttt{SEN0308\_ReadRaw} ejecuta la secuencia típica del ADC con HAL: \texttt{HAL\_ADC\_Start}, \texttt{HAL\_ADC\_PollForConversion} con un \textit{timeout} definido, lectura del valor con \texttt{HAL\_ADC\_GetValue} y parada del ADC con \texttt{HAL\_ADC\_Stop}. La función \texttt{SEN0308\_ReadRawAvg} repite esta lectura un número configurable de veces, acumula las muestras y devuelve la media (incluyendo un pequeño retardo entre muestras). Finalmente, \texttt{SEN0308\_CalculateRelative} limita la lectura al rango de calibración (\texttt{waterRaw}--\texttt{airRaw}) y aplica un mapeo lineal para obtener el porcentaje relativo, incorporando redondeo para mejorar estabilidad. \\

El proceso de uso habitual del componente sería:
\begin{enumerate}
    \item Asignar los valores de configuración del \textit{struct} \texttt{SEN0308\_t}.
    \item Llamar a \texttt{SEN0308\_Init} para inicializar el \textit{driver}.
    \item Llamar a \texttt{SEN0308\_ReadRawAvg} para obtener la lectura cruda.
    \item Convertir la lectura a humedad relativa con \texttt{SEN0308\_CalculateRelative}.
\end{enumerate}

\section{Gestión de la energía}
Para maximizar la eficiencia energética del sistema se opta por entrar en un modo de bajo consumo entre lecturas, concretamente en STOP2. En este modo el consumo es del orden de pocos microamperios según especificaciones del fabricante, lo que permite minimizar el gasto energético entre adquisiciones. \\

La contrapartida es que en STOP2 se detienen la CPU y la mayoría de relojes del sistema, por lo que al salir del modo es habitual tener que reconfigurar el reloj del sistema y reinicializar periféricos usados por la aplicación, antes de continuar con el ciclo normal. En STOP2 se conserva el contenido de SRAM y registros, y puede mantenerse activo el RTC. \\

Sin embargo el STM32WB55RG integra varios subsistemas con comportamiento de potencia independiente: la CPU1 (Cortex-M4), la CPU2 (Cortex-M0+) y el sub-sistema de radio. Cada uno puede estar en su propio estado interno, y el modo de bajo consumo “real” del sistema se alcanza únicamente cuando todos están en STOP2. Si una de las CPUs (o el radio) permanece activa, el sistema no puede caer al modo profundo y el consumo queda dominado por el subsistema que siga en funcionamiento. \\

Para salir de STOP2 en el núcleo principal existen cuatro vías:

\begin{enumerate}
    \item \textbf{RTC:} despertador periódico cada 20 minutos para leer sensores, almacenar datos y volver a STOP2.
    \item \textbf{Interrupciones externas (EXTI):} distintas alertas por pines, tratadas según su origen.
    \item \textbf{Eventos BLE/radio:} cuando el subsistema inalámbrico detecta una conexión.
    \item \textbf{UART:} usada para cargar parámetros como fecha y hora y ajustar el RTC.
\end{enumerate}

\newpage

\section{Gestión de los datos}
En la Figura \ref{fig:mem_structure} se observa la estructura a nivel de datos de la FRAM utilizada.

\begin{figure}[H]
    \centering
    \includegraphics[width=0.5\linewidth]{imgs/mem_structure.png}
    \caption{\textit{Estructura de la memoria con contenido, dirección y tamaño de los bloques. Elaboración propia.}}
    \label{fig:mem_structure}
\end{figure}

Para lograr un uso eficiente, robusto y escalable, se decidió dividir la memoria en slots de tamaño fijo, concretamente de 32 bytes. Este enfoque simplifica la gestión (direcciones y tamaños constantes), facilita la detección de errores y permite implementar mecanismos de tolerancia a fallos.

\newpage

\subsubsection{Cabecera}
En la Tabla \ref{tabla:mem_header} se expone el contenido de la cabecera guardada en el Slot 0 de la memoria.

\begin{table}[H]
    \centering
    \begin{tabular}{|c|c|l|}
        \hline
        \rowcolor{LightSteelBlue1}
        \textbf{Dirección} & \textbf{Tamaño (bytes)} & \textbf{Nombre} \\
        \hline
        \texttt{0x0000} & 8 & Información del dispositivo \\
        \hline
        \texttt{0x0008} & 8 & Reservado \\
        \hline
        \texttt{0x0010} & 8 & Metadatos de la memoria (copia A) \\
        \hline
        \texttt{0x0018} & 8 & Metadatos de la memoria (copia B) \\
        \hline
    \end{tabular}
    \caption{\textit{Distribución de la cabecera de la memoria. Elaboración propia.}}
    \label{tabla:mem_header}
\end{table}

En el bloque de información del dispositivo se almacenan datos como el identificador del dispositivo (ID) u otros campos útiles para trazabilidad o configuración. Los 8 bytes reservados se utilizan como mecanismo de verificación al arrancar: se escribe un patrón conocido, se lee de vuelta y se comprueba que coincide. Este test permite detectar problemas de comunicación con la FRAM o fallos de escritura/lectura antes de operar con datos reales. \\

La gestión de la FRAM está diseñada para soportar cortes de alimentación sin perder completamente el estado interno del almacenamiento. Para ello se guardan metadatos tales como el puntero de escritura o un contador de slots ocupados. Estos metadatos se almacenan en dos copias alternas (A y B) de 8 bytes. En cada escritura de un nuevo dato, el \textit{firmware} actualiza los metadatos, alternando entre A y B (usando un contador de secuencia). La idea es que, si hay una pérdida de alimentación durante la actualización de una de las copias, al reiniciar el sistema se pueda recuperar el estado usando la otra copia válida. Con esta estrategia, en el peor caso se pierde como máximo una muestra. \\

A continuación se observa el \textit{struct} de los metadatos guardados en memoria:

\begin{verbatim}
typedef struct {
    uint16_t write_idx; // 2 bytes
    uint16_t count;     // 2 bytes
    uint16_t crc;       // 2 bytes
    uint8_t seq;        // 1 bytes
    uint8_t commit;     // 1 bytes
} MetaFrame_t; // 8 bytes aligned
\end{verbatim}

Para asegurar que los metadatos recuperados son correctos se emplean dos mecanismos de robustez. Por un lado, se calcula y almacena un CRC (\textit{Cyclic Redundancy Check}), un código de detección de errores obtenido al aplicar un algoritmo determinista sobre los bytes almacenados. El CRC se calcula al guardar los metadatos y se vuelve a calcular al leerlos. Si durante una escritura  se corrompen uno o más bytes, lo habitual es que el CRC recalculado no coincida con el CRC almacenado. Así, durante el arranque, el sistema reconstruye el CRC a partir de los campos del \textit{struct} y lo compara con el valor guardado en el campo \texttt{crc}; si difieren, esa copia de metadatos se considera inválida y se descarta. \\

Por otro lado, se incluye un byte de \textit{commit} que actúa como indicador de escritura completada: durante la actualización de metadatos se escribe primero el contenido con \texttt{commit = 0} y, una vez finalizada correctamente la operación, se actualiza el \texttt{commit = 1}. Así, si se interrumpe la alimentación a mitad del proceso, el \textit{commit} no queda validado y esa copia se descarta, recurriendo a la otra copia (A o B) que sí sea coherente. \\

En el caso de que ambas copias sean inválidas, el sistema considera que no existe un estado previo recuperable y realiza una inicialización limpia de los metadatos en la copia A.

\subsubsection{Datos}
El resto de la memoria (del slot 1 al 1023) se destina al almacenamiento de muestras de medida. El paquete almacenado incorpora, además de los valores medidos, mecanismos de robustez similares a los usados en los metadatos: un CRC para detección de corrupción y un byte de commit para detectar escrituras incompletas. \\

La alineación y el orden de los campos fue un aspecto importante del diseño. Al trabajar con un tamaño fijo de slot, era necesario garantizar que el \textit{frame} completo ocupase exactamente 32 bytes y que el compilador no introdujese rellenos inesperados. Por ello, los campos se ordenaron conscientemente y se añadieron bytes reservados al final para mantener un tamaño constante. \\

Además se incorpora un vector de bits que indica qué medidas de la muestra son válidas.

\begin{verbatim}
typedef struct {
    float irradiance_Wm2;               // 4 byte
    float airTemp_C;                    // 4 bytes
    float soilTemp_C;                   // 4 bytes
    uint8_t airHumidity_perc;           // 1 bytes
    uint8_t soilMoisture_perc;          // 1 byte
    uint16_t batteryVoltage_mV;         // 2 bytes
    uint8_t hours, minutes, seconds;    // 3 bytes
    uint8_t day, month, year;           // 3 bytes
    uint16_t validDataVector;           // 2 bytes
} DataSample_t; // 24 bytes aligned

typedef struct {
    DataSample_t data;      // 24 bytes
    uint16_t crc;           // 2 bytes
    uint8_t commit;         // 1 bytes
    uint8_t  _reserved[5];  // 5 bytes
} DataFrame_t; // 32 bytes aligned
\end{verbatim}

Para garantizar que el tamaño de los \textit{frames} coincide exactamente con los tamaños esperados, se añadieron las comprobaciones en tiempo de compilación siguientes:

\begin{verbatim}
_Static_assert(sizeof(DataFrame_t) == 32, "DataFrame_t must be 32 bytes");
_Static_assert(sizeof(MetaFrame_t) == 8, "MetaFrame_t must be 8 bytes");
\end{verbatim}

\chapter{Validación y pruebas}
Una vez desarrollados el \textit{hardware} y el \textit{firmware} se inició el periodo de verificación experimental del prototipo completo, con el objetivo de comprobar el funcionamiento del sistema en condiciones semi-reales de uso y validar los aspectos críticos del diseño.

\section{Pruebas de memoria}
Se diseñaron pruebas para validar tanto la lectura/escritura básica como el comportamiento del sistema en escenarios límite. En todos los casos el objetivo es asegurar que el registro de datos es consistente y que el sistema puede operar durante periodos prolongados sin degradación del almacenamiento. \\

Como metodología común, las pruebas se ejecutaron mediante un bucle periódico en el que el \textit{firmware} realiza: (1) escritura de un \textit{frame} en FRAM, (2) lectura inmediata del mismo \textit{slot} y (3) comprobación de validez del registro leído.

\subsubsection{Escritura y lectura básicas}
Como prueba inicial se implementó un ensayo en el que el sistema escribe un registro en FRAM y lo lee de vuelta a continuación. Esta prueba permite detectar fallos de comunicación SPI, errores de direccionamiento y problemas de coherencia durante escritura. \\

Criterio de aceptación: la lectura posterior debe coincidir con lo escrito durante un número elevado de ciclos consecutivos, sin errores ni valores inconsistentes. \\

En el \textit{log} se observa que, tras cada escritura, la lectura del mismo \textit{slot} es correcta y el registro se marca como válido (\texttt{Read Valid: 1}). Además, durante la inicialización se detecta que ambas copias de metadatos son válidas (\texttt{Both meta valid}), lo cual confirma que el sistema puede recuperar el contexto previamente almacenado.

\newpage

\scriptsize
\begin{verbatim}
Both meta valid
FRAM inicializada correctamente

Write Slot: 0 (0x0020)
Write Count: 1
Write Meta B
Read Slot: 0 (0x0020)
Read Valid: 1

Write Slot: 1 (0x0040)
Write Count: 2
Write Meta A
Read Slot: 1 (0x0040)
Read Valid: 1

Write Slot: 2 (0x0060)
Write Count: 3
Write Meta B
Read Slot: 2 (0x0060)
Read Valid: 1
...
\end{verbatim}
\normalsize

\subsubsection{Prueba de \textit{overflow}}
En esta prueba se valida el comportamiento del sistema cuando el área de datos alcanza su capacidad máxima. El prototipo implementa una política de \textit{buffer} circular, de forma que al llenarse la memoria el sistema continúa almacenando nuevas muestras sobrescribiendo progresivamente las más antiguas. \\

Criterio de aceptación: al alcanzar el número máximo de muestras, el sistema debe seguir operando sin bloquearse y continuar escribiendo registros válidos de forma controlada. \\

En el \textit{log} se observa que al llegar al valor máximo del contador (\texttt{Write Count: 1023}) el sistema continúa ejecutando el ciclo de escritura/lectura y mantiene lecturas válidas. Esto confirma que el sistema no se detiene al llenarse la FRAM y que el almacenamiento sigue siendo funcional.

\scriptsize
\begin{verbatim}
...
Write Slot: 304 (0x2620)
Write Count: 1022
Write Meta B
Read Slot: 304 (0x2620)
Read Valid: 1

Write Slot: 305 (0x2640)
Write Count: 1023
Write Meta A
Read Slot: 305 (0x2640)
Read Valid: 1

Write Slot: 306 (0x2660)
Write Count: 1023
Write Meta B
Read Slot: 306 (0x2660)
Read Valid: 1

Write Slot: 307 (0x2680)
Write Count: 1023
Write Meta A
Read Slot: 307 (0x2680)
Read Valid: 1
...
\end{verbatim}
\normalsize

\subsubsection{Prueba de \textit{reset} lógico de FRAM y recuperación tras reinicio}
Esta prueba evalúa dos escenarios: (1) un \textit{reset} lógico del subsistema de memoria (reinicialización del área de datos y metadatos) y (2) un reinicio externo del microcontrolador, para comprobar que no se pierde el contexto al arrancar. En el ensayo se fuerza un \textit{reset} lógico cada 10 escrituras. \\

Criterio de aceptación: tras el \textit{reset} lógico el sistema debe reiniciar el conteo/índice de escritura sin errores, y tras un reinicio externo debe recuperar el contexto desde los metadatos almacenados y continuar donde lo dejó. \\

En el \textit{log} se aprecia que, tras el reset lógico (\texttt{Reset FRAM}), el sistema vuelve a escribir desde el \textit{slot} 0 con contador reiniciado. Asimismo, tras un \texttt{<Reset externo>} se observa la detección de metadatos válidos (\texttt{Both meta valid}) y la continuación del conteo sin pérdida de contexto.

\scriptsize
\begin{verbatim}
...
Write Slot: 9 (0x0140)
Write Count: 10
Write Meta A
Read Slot: 9 (0x0140)
Read Valid: 1

Reset FRAM

Write Slot: 0 (0x0020)
Write Count: 1
Write Meta B
Read Slot: 0 (0x0020)
Read Valid: 1

Write Slot: 1 (0x0040)
Write Count: 2
Write Meta A
Read Slot: 1 (0x0040)
Read Valid: 1

<Reset externo>

Both meta valid
FRAM inicializada correctamente

Write Slot: 2 (0x0060)
Write Count: 3
Write Meta B
Read Slot: 2 (0x0060)
Read Valid: 1
...
\end{verbatim}
\normalsize

\subsubsection{Prueba de borrado completo}
En esta prueba se ejecuta un borrado completo de la FRAM y se verifica que la memoria queda con todos los bytes a 0. Tras el borrado se reinicializa el subsistema y se comprueba que el sistema puede volver a operar con normalidad. \\

Criterio de aceptación: tras ejecutar el borrado completo, la memoria debe considerarse vacía (sin metadatos válidos), y el sistema debe ser capaz de reinicializarse y comenzar a almacenar desde el inicio sin errores. \\

Para asegurar que el borrado se ha realizado correctamente, el \textit{firmware} recorre la región borrada y comprueba que todos los bytes leídos son \texttt{0x00}. Si se detectase cualquier byte distinto de cero, el sistema generaría un mensaje de error y marcaría el borrado como fallido, evitando continuar con la memoria en un estado inconsistente. \\

En el \textit{log} se observa que antes del borrado el sistema puede detectar metadatos válidos (\texttt{Both meta valid}), mientras que tras ejecutar \texttt{Erase FRAM} pasa a \texttt{None meta valid}, comportamiento coherente con una memoria limpia. A continuación el sistema se reinicializa y comienza a escribir desde el \textit{slot} 0 con lecturas válidas.

\scriptsize
\begin{verbatim}
<Reset externo>

Both meta valid
FRAM inicializada correctamente

Erase FRAM

None meta valid
FRAM inicializada correctamente

Write Slot: 0 (0x0020)
Write Count: 1
Write Meta B
Read Slot: 0 (0x0020)
Read Valid: 1

Write Slot: 1 (0x0040)
Write Count: 2
Write Meta A
Read Slot: 1 (0x0040)
Read Valid: 1
...
\end{verbatim}
\normalsize

\section{Pruebas de autonomía}
Para evaluar el balance energético del sistema durante un funcionamiento continuo se diseñó una prueba que consistió en dejar el prototipo operando durante 24 horas completas. Para aumentar la representatividad del ensayo, se utilizó el periodo de muestreo objetivo de 20 min y se almacenaron todas las muestras en memoria. \\

El ensayo se realizó en interior, cerca de una ventana para maximizar la exposición a la luz disponible. Además, la prueba se llevó a cabo en invierno y en un día nublado, por lo que las condiciones de captación son desfavorables frente a un despliegue real en exterior. En consecuencia, los resultados obtenidos pueden interpretarse como una cota conservadora del comportamiento esperado. \\

Durante el experimento se tomaron 73 muestras, desde las 10:45 hasta las 10:45 del día siguiente, con todos los \textit{slots} marcados como válidos, lo que indica que el sistema mantuvo el ciclo de adquisición y almacenamiento sin interrupciones. \\

\subsubsection{Resultados}
La Figura \ref{fig:bateria_irradiancia} muestra la evolución temporal de la tensión de batería y la irradiancia estimada durante todo el ensayo.

\begin{figure}[H]
    \centering
    \includegraphics[width=1\linewidth]{imgs/bateria_irradiancia.png}
    \caption{\textit{Evolución de la tensión de batería e irradiancia estimada durante 24 h. Elaboración propia.}}
    \label{fig:bateria_irradiancia}
\end{figure}

En primer lugar, se observa que la tensión final es inferior a la inicial. La tensión de batería al inicio del ensayo fue de $V_\mathrm{ini}=3.832~\mathrm{V}$ y al final de $V_\mathrm{fin}=3.768~\mathrm{V}$, lo que corresponde a una caída neta de $\Delta V=-0.064~\mathrm{V}=-64~\mathrm{mV}$ en 24 horas. \\

Esta descarga neta está fuertemente influenciada por las condiciones desfavorables del experimento (interior, invierno y cielo nublado). Aun así, el comportamiento durante los picos de irradiancia muestra que la batería es capaz de recuperar tensión con rapidez: el máximo de irradiancia registrado fue de 51.6 $\mathrm{W/m^2}$ a las 13:05, coincidiendo con un máximo de batería de $V_\mathrm{max}=3.872~\mathrm{V}$. Si niveles de irradiancia similares se mantuvieran durante un intervalo más prolongado, como ocurriría en una instalación real en exterior, sería esperable una carga diaria mayor y, por tanto, un balance energético más favorable. \\

Adicionalmente, se aprecia que a partir de las 18:05 la irradiancia pasa a 0 $\mathrm{W/m^2}$ y se mantiene prácticamente nula durante la noche, intervalo en el que la tensión desciende de forma progresiva. Cabe recalcar que los valores de irradiancia del ensayo son reducidos en comparación con magnitudes típicas en exterior incluso en días nublados, los cuales podrían rondar los 100 $\mathrm{W/m^2}$ en estas mismas condiciones. \\

Aun con estas condiciones, la caída de tensión observada es baja: 64 mV en 24 h. Si se realizase una extrapolación lineal, considerando una batería totalmente cargada en torno a $V_\mathrm{OVCH}\approx 4.12~\mathrm{V}$ y un umbral de descarga $V_\mathrm{OVDIS}=3.6~\mathrm{V}$, el tiempo para alcanzar dicho umbral sería:

\[
t \approx \frac{V_\mathrm{OVCH}-V_\mathrm{OVDIS}}{0.064}\approx \frac{4.12-3.6}{0.064}\approx 8.1~\mathrm{días}
\]

En conclusión, aunque el ensayo presenta un balance neto ligeramente negativo en un escenario conservador, el comportamiento observado sugiere que en un entorno real el sistema debería mejorar de forma significativa su balance energético y, por tanto, su autonomía efectiva.

\section{Pruebas de sensores}
A continuación se realizaron una serie de experimentos para evaluar el comportamiento de los sensores integrados en el prototipo.

\subsubsection{Consistencia de temperatura y humedad durante 24 h}
Con el objetivo de validar la coherencia de las medidas en funcionamiento continuo, se reutilizaron los datos registrados durante el experimento de autonomía. En este ensayo, ambos sensores de temperatura (SHT31 y DS18B20) se encontraban midiendo en aire, por lo que sus lecturas deberían ser comparables. Como se observa en la Figura \ref{fig:temp_hum}, ambas curvas siguen una tendencia muy similar durante todo el periodo, con diferencias pequeñas, lo que constituye una buena señal del correcto funcionamiento de ambos sensores y de su calibración relativa. \\

En cuanto a la humedad relativa medida por el SHT31, se mantiene estable dentro de un rango acotado, oscilando entre el 37\% y el 46\% durante las 24 horas. Además, se aprecia un comportamiento coherente con el ciclo día y noche: la humedad relativa tiende a ser menor durante las horas de mayor iluminación y aumenta ligeramente durante la noche.

\begin{figure}[H]
    \centering
    \includegraphics[width=1\linewidth]{imgs/temp_hum.png}
    \caption{\textit{Evolución de temperatura y humedad relativa durante el ensayo de 24 h. Elaboración propia.}}
    \label{fig:temp_hum}
\end{figure}

\subsubsection{Pruebas de humedad relativa ante cambios bruscos}
Además del análisis de 24 h, se realizó una prueba específica para evaluar la respuesta temporal del sensor de humedad relativa (SHT31) ante cambios rápidos del entorno. Para ello se configuró un muestreo rápido, tomando una medida cada 2 s, y se generaron dos incrementos forzados de humedad aplicando vapor de agua cerca del sensor. \\

En la Figura \ref{fig:humedad_relativa} se observa que el sensor responde de forma clara ante ambos estímulos, alcanzando valores cercanos a la saturación y posteriormente recuperándose hacia el nivel base. La prueba se repitió dos veces y se aprecia un comportamiento distinto en la fase de recuperación: en el primer pico la humedad decrece de forma más lenta, mientras que en el segundo la caída es considerablemente más rápida. Esta diferencia se atribuye a las condiciones de ventilación: durante el primer pico apenas había circulación de aire, mientras que en el segundo sí existía ventilación, acelerando la renovación del aire y, por tanto, la recuperación de la humedad relativa hacia el valor inicial.

\begin{figure}[H]
    \centering
    \includegraphics[width=1\linewidth]{imgs/humedad_relativa.png}
    \caption{\textit{Respuesta temporal del sensor de humedad relativa ante dos picos forzados de humedad. Elaboración propia.}}
    \label{fig:humedad_relativa}
\end{figure}

\newpage

\subsubsection{Pruebas de humedad en terreno}
Para validar el sensor de humedad de suelo se realizó un ensayo sencillo en una maceta, clavando el sensor en el sustrato y registrando simultáneamente la lectura de humedad (SEN0308) y la temperatura de suelo (DS18B20). El objetivo fue obtener una calibración absoluta, sino comprobar que el sensor responde de forma clara ante cambios de humedad y que el sistema es capaz de registrar dicha variación de manera estable.

\begin{figure}[H]
    \centering
    \includegraphics[width=0.8\linewidth]{imgs/maceta.jpg}
    \caption{\textit{Montaje del ensayo de humedad de suelo con los sensores insertado en una maceta. Elaboración propia.}}
    \label{fig:maceta}
\end{figure}

La Figura \ref{fig:maceta} muestra el montaje del experimento. Previamente, para disponer de una conversión aproximada a porcentaje, se realizó una calibración por dos puntos extremos: lectura en aire como referencia de 0\% (seco) y lectura con el sensor completamente sumergido en agua como referencia de 100\% (húmedo). A partir de estos dos valores se extrapoló linealmente el porcentaje para las medidas intermedias en la maceta. \\

La evolución temporal de las medidas se muestra en la Figura \ref{fig:hum_temp_terreno}. Al inicio del experimento se añadieron aproximadamente 100 mL de agua a la maceta, lo que provocó un incremento repentino en la humedad del suelo. Tras este aporte, la humedad descendió lentamente hasta aproximadamente las 12:00, momento a partir del cual el sol incide directamente sobre la maceta. Esta exposición directa acelera la evaporación y se observa una caída más rápida de la humedad medida. Cuando finaliza la incidencia directa del sol, la humedad se recupera parcialmente y tiende a estabilizarse alrededor del 50\%. Cabe recalcar que el experimento se llevó a cabo en exterior y en invierno, lo que en general dificulta la evaporación del agua y hace que la evolución temporal observada sea conservadora.

\begin{figure}[H]
    \centering
    \includegraphics[width=1\linewidth]{imgs/hum_temp_terreno.png}
    \caption{\textit{Evolución de la humedad y la temperatura de suelo durante el experimento en la maceta. Elaboración propia.}}
    \label{fig:hum_temp_terreno}
\end{figure}

En cuanto a la temperatura, se observa una perturbación en el instante en el que se añade el agua, atribuible a la redistribución térmica en el terreno: la tierra se encontraba fría tras haber permanecido expuesta durante la noche y, al añadir el agua, la temperatura medida aumenta de forma puntual. Posteriormente, la temperatura asciende progresivamente hasta aproximadamente las 11:30, donde comienza a descender. Cuando el sol incide directamente sobre la maceta, se aprecia un ligero incremento adicional, tras el cual la temperatura vuelve a disminuir gradualmente hasta el final del ensayo. \\

Finalmente, conviene destacar que esta calibración (aire como 0\% y agua como 100\%) es orientativa y depende del tipo de suelo y sus propiedades. Por tanto, la medida debe interpretarse como un indicador relativo del estado hídrico del terreno a menos que se realice una calibración específica para el sustrato de interés.

\section{Pruebas de conexión inalámbrica}
Debido a limitaciones de tiempo, no se integró la funcionalidad BLE completa en el \textit{firmware} final del prototipo. No obstante, se validó el correcto funcionamiento del subsistema inalámbrico del microcontrolador mediante ejemplos de código proporcionados por STMicroelectronics, en concreto se uso el \texttt{BLE\_p2pServer}. Este \textit{firmware} de demostración implementa un servicio BLE sencillo de tipo \textit{peer-to-peer}, con características para el intercambio básico de datos entre el dispositivo y un \textit{cliente} (en nuestro caso un teléfono móvil), permitiendo comprobar de forma aislada la pila BLE y el funcionamiento de la radio. \\

Durante la validación se comprobó que el dispositivo:
\begin{itemize}
    \item Inicia correctamente el \textit{advertising} y es detectable desde un \textit{scanner} BLE.
    \item Permite establecer conexión de manera estable.
    \item Permite el intercambio de datos básico mediante el servicio \textit{p2p} (lectura/escritura/notify).
\end{itemize}

Como resultado, se considera validada la operatividad del \textit{hardware} de radio y la viabilidad de incorporar en el futuro el volcado de datos almacenados en FRAM mediante un servicio BLE, quedando dicha integración fuera del alcance del presente trabajo.

\chapter{Conclusiones y trabajo futuro}
Este Trabajo de Fin de Grado ha consistido en el diseño y desarrollo de un prototipo de una lanza meteorológica para la monitorización ambiental, abarcando el ciclo completo del proyecto: definición del sistema, selección de componentes, diseño de la electrónica, fabricación y puesta en marcha, desarrollo del \textit{firmware} y pruebas iniciales de verificación. \\

En relación con los requerimientos planteados, el prototipo cumple los objetivos principales de adquisición y almacenamiento local, además de un funcionamiento periódico utilizando el RTC, con lectura de sensores y registro en la FRAM. A nivel de diseño físico, la PCB se ha desarrollado teniendo en cuenta las restricciones del fabricante y buenas prácticas de ruteado y distribución por capas, obteniendo una plataforma integrable en la lanza y preparada para evolucionar. \\

Los resultados obtenidos en las pruebas controladas muestran un comportamiento coherente de la sensórica y del sistema de registro. En particular, se ha verificado la lectura periódica y el almacenamiento consistente de las muestras, así como la respuesta esperada de los sensores ante cambios en el entorno. Además, se ha validado el funcionamiento básico del subsistema inalámbrico, confirmando que la plataforma \textit{hardware} permite incorporar comunicación BLE como canal de volcado, aunque dicha integración queda fuera del alcance actual. \\

Desde el punto de vista de utilidad, el trabajo aporta a la empresa Domo21 una base técnica concreta y documentada sobre la que construir una solución de monitorización en campo: una arquitectura validada, una PCB propia fabricable, y un \textit{firmware} modular que ya resuelve el ciclo de medida y el registro local. Para mí, el proyecto ha supuesto una experiencia completa de desarrollo de un sistema embebido real, integrando diseño electrónico, consideraciones de fabricación, gestión de bajo consumo, estructuración de datos y validación experimental, con un resultado tangible y reutilizable.

\subsubsection{Trabajo futuro}
Como continuación natural del proyecto, el primer paso sería completar la parte de la comunicación inalámbrica, implementando el BLE de forma completa en el \textit{firmware}, permitiendo el volcado de los datos almacenados en la FRAM a un dispositivo externo. \\

En segundo lugar, el prototipo necesita tiempo de campo. Las pruebas en laboratorio han servido para comprobar el comportamiento básico, pero una validación real exige semanas de operación continua, con días buenos y malos, y cambios de temperatura, humedad y radiación solar. Estas campañas permitirían verificar la autonomía en condiciones representativas, observar la estabilidad de las medidas a lo largo del tiempo y descubrir detalles prácticos que solo aparecen al desplegar el sistema. \\

Finalmente, se plantearía integrar el sistema como parte del proyecto de monitorización con drones, evolucionando el prototipo hacia una solución más completa. Con la experiencia obtenida, sería posible desarrollar una versión mejorada y más funcional, alineada con el objetivo final del proyecto.
