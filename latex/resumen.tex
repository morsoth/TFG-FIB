\clearpage
\thispagestyle{empty}
\null
\clearpage

\Large
\textbf{Agradecimientos} \\
\normalsize

Quiero agradecer primeramente a mi director del trabajo Antonio Benito Martínez, así como a Carlos Morata y Álex Veiga, por ayudarme a realizar mi TFG tanto en aspectos técnicos como metodológicos. También me gustaría agradecer al departamento de ESAII de la UPC y a la empresa Domo21 por aportar los recursos necesarios para llevar a cabo todo el proyecto. \\

Por último, agradecer a mi familia, especialmente a mi madre y hermano, por su apoyo incondicional y por ayudarme a redactar la memoria de este trabajo. Muchas gracias.

\newpage

\Large
\textbf{Resumen} \\
\normalsize

Este trabajo se centra en el diseño y desarrollo de un prototipo de \textbf{lanza meteorológica} para la monitorización ambiental en campo, dentro de un proyecto más amplio de adquisición de datos con drones. El proyecto nace como una mejora frente a soluciones más aparatosas o dependientes de conectividad, resolviendo sobre todo la necesidad de contar con un nodo compacto, autónomo y fiable para campañas en exteriores. El objetivo principal es \textbf{diseñar y fabricar una PCB} capaz de registrar variables ambientales de forma periódica y guardar los datos localmente durante largos periodos, incorporando una gestión energética orientada a bajo consumo y una arquitectura modular que facilite añadir o sustituir sensores en el futuro. \\

\Large
\textbf{Resum} \\
\normalsize

Aquest treball se centra en el disseny i el desenvolupament d’un prototip de \textbf{llança meteorològica} per a la monitorització ambiental en camp, dins d’un projecte més ampli d’adquisició de dades amb drons. El projecte neix com una millora davant de solucions més aparatosses o dependents de connectivitat, resolent sobretot la necessitat de disposar d’un node compacte, autònom i fiable per a campanyes a l’exterior. L’objectiu principal és \textbf{dissenyar i fabricar una PCB} capaç de registrar variables ambientals de manera periòdica i desar les dades localment durant llargs períodes, incorporant una gestió energètica orientada a baix consum i una arquitectura modular que faciliti afegir o substituir sensors en el futur. \\

\Large
\textbf{Abstract} \\
\normalsize

This project focuses on the design and development of a \textbf{meteorological probe} prototype for in-field environmental monitoring, as part of a broader drone-based data acquisition initiative. It is conceived as an improvement over bulkier solutions or systems that depend on continuous connectivity, addressing the need for a compact, autonomous, and reliable node for outdoor campaigns. The main goal is to \textbf{design and build a PCB} capable of periodically logging environmental variables and storing the data locally for long periods, combining low-power energy management with a modular architecture that makes it easy to add or replace sensors in the future. \\